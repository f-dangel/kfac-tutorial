% ===================================================================
% WRITING
% ===================================================================
\usepackage{paracol}
\usepackage{blindtext}

% ===================================================================
% COLORS
% ===================================================================
% VECTOR PRIMARY COLORS
\definecolor{VectorBlack}{RGB}{34, 34, 34}
\definecolor{VectorGray}{RGB}{239, 238, 237}

% VECTOR SECONDARY COLORS
\definecolor{VectorBlue}{RGB}{59, 69, 227}
\definecolor{VectorPink}{RGB}{253, 8, 238}
\definecolor{VectorOrange}{RGB}{250, 173, 26}
\definecolor{VectorTeal}{RGB}{82, 199, 222}

% ===================================================================
% REFERENCES
% ===================================================================
\hypersetup{%
  colorlinks,
  citecolor = VectorBlue,%
  linkcolor = VectorBlue,%
  urlcolor = VectorPink,%
}%
% tell cleveref how to name a listing environment
\crefname{listing}{snippet}{snippets}
% Rename Listing into Snippet
\renewcommand{\lstlistingname}{Snippet}

% ===================================================================
% CODE BLOCKS
% ===================================================================
% define style for listings using the Vector color scheme
\lstdefinestyle{vector_institute}{
  backgroundcolor=\color{VectorGray!50},
  commentstyle=\bfseries\color{VectorBlue},
  keywordstyle=\bfseries\color{VectorBlack},
  numberstyle=\tiny\color{VectorBlack!50},
  stringstyle=\bfseries\color{VectorPink},
  basicstyle=\ttfamily\scriptsize,
  xleftmargin=3.2ex,
  breakatwhitespace=false,
  breaklines=true,
  captionpos=t,
  keepspaces=true,
  numbers=left,
  numbersep=7pt,
  showspaces=false,
  showstringspaces=false,
  showtabs=false,
  tabsize=2,
  % interpret anything between @...@ in a .py file as LaTeX code
  escapeinside={(@}{@)},
  escapebegin={\bfseries\color{blue}},
  % increase block width to reduce number of lines
  linewidth=\linewidth,
}
% use the above style as default
\lstset{style=vector_institute}

\usepackage{caption}
% modify captions of code blocks
\captionsetup[lstlisting]{%
  font={scriptsize},%
  justification=raggedright,%
  singlelinecheck=false,%
}

\newcommand{\repourl}{https://github.com/f-dangel/kfac-from-scratch}
% Command to include code blocks and their output.
% First argument is the filename inside the kfac_tutorial directory.
% The listing can be referenced with that filename, too.
\newcommand{\codeblockWithOutput}[1]{
  % convert _ into \_ for LaTeX
  \def\filename{\detokenize{#1}}
  \lstinputlisting[%
  language=python,%
  caption=\href{\repourl/kfs/#1.py}{\texttt{kfs/\filename.py}},%
  label=#1,%
  ]{../kfs/#1.py}
  \vspace{-1.25\baselineskip}
  \lstinputlisting[%
  language=python,%
  numbers=none,%
  basicstyle=\bfseries\ttfamily\scriptsize,%
  title=\href{\repourl/tex/output/#1.txt}{Output:}\hfill,%
  ]{output/#1.txt}
}

% ===================================================================
% MATH
% ===================================================================
\theoremstyle{definition}
\newtheorem{caveat}[theorem]{Caveat}
\DeclareMathOperator{\rvec}{rvec}
\DeclareMathOperator{\cvec}{cvec}
%%% Local Variables:
%%% mode: latex
%%% TeX-master: "../main"
%%% End:
