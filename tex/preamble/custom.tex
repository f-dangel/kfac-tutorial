% ===================================================================
% WRITING
% ===================================================================
\usepackage{comment}
\usepackage{paracol}
\usepackage{blindtext}
\newcommand{\felix}[1]{\textcolor{red}{[\textbf{Felix:} #1]}}
\newcommand{\balint}[1]{\textcolor{orange}{Bálint: #1}}
\usepackage{xspace}
\newcommand*{\ie}{i.e.\@\xspace}
\newcommand*{\iid}{i.i.d.\@\xspace}
\newcommand*{\wrt}{w.r.t.\@\xspace}
\newcommand*{\eg}{e.g.\@\xspace}
\newcommand*{\Ie}{I.e.\@\xspace}
\newcommand*{\Eg}{E.g.\@\xspace}

% ===================================================================
% COLORS
% ===================================================================
% VECTOR PRIMARY COLORS
\definecolor{VectorBlack}{RGB}{34, 34, 34}
\definecolor{VectorGray}{RGB}{239, 238, 237}

% VECTOR SECONDARY COLORS
\definecolor{VectorBlue}{RGB}{59, 69, 227}
\definecolor{VectorPink}{RGB}{253, 8, 238}
\definecolor{VectorOrange}{RGB}{250, 173, 26}
\definecolor{VectorTeal}{RGB}{82, 199, 222}
\newcommand{\colored}[2][VectorBlue]{{\color{#1}#2}}

% ===================================================================
% REFERENCES
% ===================================================================
\hypersetup{%
  colorlinks,
  citecolor = VectorBlue,%
  linkcolor = VectorBlue,%
  urlcolor = VectorPink,%
}%
% tell cleveref how to name a listing environment
\crefname{listing}{snippet}{snippets}
% Rename Listing into Snippet
\renewcommand{\lstlistingname}{Snippet}

% ===================================================================
% CODE BLOCKS
% ===================================================================
% define style for listings using the Vector color scheme
\lstdefinestyle{vector_institute}{
  backgroundcolor=\color{VectorGray!50},
  commentstyle=\bfseries\color{VectorBlue},
  keywordstyle=\bfseries\color{VectorBlack},
  numberstyle=\tiny\color{VectorBlack!50},
  stringstyle=\bfseries\color{VectorBlue},
  basicstyle=\ttfamily\scriptsize,
  xleftmargin=3.2ex,
  breakatwhitespace=false,
  breaklines=true,
  captionpos=t,
  keepspaces=true,
  numbers=left,
  numbersep=7pt,
  showspaces=false,
  showstringspaces=false,
  showtabs=false,
  tabsize=2,
  escapebegin={\color{VectorBlue}},
  linewidth=\linewidth,
  mathescape=true,
}
% use the above style as default
\lstset{style=vector_institute}

\usepackage{caption}
% modify captions of code blocks
\captionsetup[lstlisting]{%
  font={scriptsize},%
  justification=raggedright,%
  singlelinecheck=false,%
}

% URL with branch/commit that should be used to reference the code
\newcommand{\repourl}{https://github.com/f-dangel/kfac-tutorial/blob/main}
\newcommand{\repofile}[1]{
  % convert _ into \_ for LaTeX using \detokenize
  \href{\repourl/kfs/#1.py}{\texttt{kfs/\detokenize{#1}.py}}
}
% Command to include code blocks and their output.
% First argument is the filename inside the kfac_tutorial directory.
% The listing can be referenced with that filename, too.
\newcommand{\codeblock}[1]{
  \lstinputlisting[%
  language=python,%
  caption={\repofile{#1}},%
  label=#1,%
  ]{../kfs/#1.py}
}

% ===================================================================
% MATH
% ===================================================================
\usepackage{nicefrac}
\DeclareMathOperator{\rvec}{rvec}
\DeclareMathOperator{\cvec}{cvec}
\let\vec\relax % delete the existing \vec command
\DeclareMathOperator{\vec}{vec}
\DeclareMathOperator{\mat}{mat}
\DeclareMathOperator{\lin}{lin}

\usepackage{mdframed}
\mdfdefinestyle{custom}{%
  linecolor=black,%
  topline=false,%
  bottomline=false,%
  rightline=false,%
  linewidth=1.25pt,%
  backgroundcolor=VectorGray!50,%
  innerleftmargin=5pt,%
  % innertopmargin=8pt,%
}
\theoremstyle{definition}
\newmdtheoremenv[style=custom]{definition}{Definition}[section]
\newmdtheoremenv[style=custom]{setup}{Setup}[section]
\newmdtheoremenv[%
style=custom,%
linecolor=VectorOrange,%
backgroundcolor=VectorOrange!10,%
]{caveat}{Caveat}[section]
\newmdtheoremenv[%
style=custom,%
linecolor=VectorTeal,%
backgroundcolor=VectorTeal!10,%
]{test}{Test}[section]
\newmdtheoremenv[%
style=custom,%
linecolor=VectorTeal,%
backgroundcolor=VectorTeal!10,%
]{example}{Example}[section]

% ===================================================================
% FIGURES
% ===================================================================
\usepackage{tikz}
\usetikzlibrary{arrows.meta}
\usetikzlibrary{positioning}

%%% Local Variables:
%%% mode: latex
%%% TeX-master: "../main"
%%% End:
