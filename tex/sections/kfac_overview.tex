\switchcolumn[1]*
\codeblock{kfac}
\switchcolumn[0]

Generally speaking, KFAC approximates a matrix of the form
\begin{align*}
  \mC(\vtheta^{(i)})
  = R \sum_n
  (\jac_{\vtheta^{(i)}} \vf_n)^{\top}
  \left[ \bullet(\vf_n, \vy_n) \right]
  (\jac_{\vtheta^{(i)}} \vf_n)
\end{align*}
where $\bullet \in \sR^{\gF \times \gF}$ is a positive semi-definite matrix in the prediction space that depends on the prediction and target, and which is sandwiched between the Jacobians of the network's prediction w.r.t.\,parameters $\vtheta^{(i)}$ in a layer $i$.
Many different matrices satisfy the above structure: the GGN/type-II Fisher, the type-I Fisher, and the empirical Fisher.
The goal is to approximate $\mC(\vtheta^{(i)}$ with a Kronecker product,
\begin{align*}
  \mC(\vtheta^{(i)})
  \approx
  \mA^{(i)} \otimes \mB^{(i)}
  \coloneq
  \kfac(\mC(\vtheta^{(i)}))
\end{align*}
$\mA^{(i)}$ will be computed from the inputs to layer $i$, therefore we will refer to it as \emph{input-based Kronecker factor}.
$\mB^{(i)}$ will be computed from gradients w.r.t.\,layer $i$'s output, therefore we will refer to it as the \emph{grad-output-based Kronecker factor}.
More intuition about this in the following paragraph.

\paragraph{Why use a Kronecker structure?}
Despite the computational benefits of using a Kronecker-structured approximation (cheap storage and inversion), this structure naturally occurs in the above expression.
To see this, let's re-write $\mC(\vtheta^{(i)})$ a little.
First, since $\bullet(\vf_n, \vy_n)$ is PSD, we can express it as an outer product of $\dim(\gF)$ vectors,
$\bullet(\vf_n, \vy_n) = \sum_{c} \blacktriangle_c(\vf_n, \vy_n) (\blacktriangle_c(\vf_n, \vy_n))^{\top}$ where $\blacktriangle_c(\vf_n, \vy_n) \in \sR^{\gF}$ (we use the shorter $\blacktriangle_{n,c}$ below. This turns the curvature matrix into
\begin{align*}
  \mC(\vtheta^{(i)})
  =
  R \sum_n \sum_{c}
  (\jac_{\vtheta^{(i)}} \vf_n)^{\top}
  \blacktriangle_{n,c} \blacktriangle_{n,c}^{\top}
  (\jac_{\vtheta^{(i)}} \vf_n)\,.
\end{align*}
Next, we will use the chain rule to split the Jacobian at the layer output $\vx^{(i)} = f^{(i)}(\vx^{(i-1)}, \vtheta^{(i)})$,
\begin{align*}
  (\jac_{\vtheta^{(i)}} \vf_n)^{\top}
  =
  (\jac_{\vtheta^{(i)}} \vx^{(i)}_n)^{\top}
  (\jac_{\vx_n^{(i)}} \vf_n)^{\top}\,.
\end{align*}
Inserting this, we obtain
\begin{align*}
  \mC(\vtheta^{(i)})
  =
  R \sum_n \sum_{c}
  (\jac_{\vtheta^{(i)}} \vx^{(i)}_n)^{\top}
  (\jac_{\vx_n^{(i)}} \vf_n)^{\top}
  \blacktriangle_{n,c}
  \\
  \blacktriangle_{n,c}^{\top}
  (\jac_{\vx_n^{(i)}} \vf_n)
  (\jac_{\vtheta^{(i)}} \vx^{(i)}_n)
\end{align*}
This form now makes it much easier to understand why there is an layer input-based Kronecker factor and a grad-layer-output based Kronecker factor: We can think of $(\jac_{\vtheta^{(i)}} \vx_n^{(i)})^{\top} \blacktriangle_{n,c}$ as a pseudo-gradient.
In fact, if we set $\blacktriangle_{n,c} \leftarrow \nabla_{\vf_n} c(\vf_n, \vy_n)$, we obtain $(\jac_{\vtheta^{(i)}} \vx_n^{(i)})^{\top} \blacktriangle_{n,c} = \nabla_{\vx_n^{(i)}} c(\vf_n, \vy_n)$, i.e.\,the per-datum loss gradient w.r.t.\,layer $i$'s output.
The Kronecker structure emerges in the output-parameter Jacobian $\jac_{\vtheta^{(i)}} \vx_n^{(i)}$ of a layer.
We already encountered this in previous sections, e.g.\,the output-weight Jacobian of a linear layer has Kronecker structure (\Cref{jacobians_linear_layer}) where one of the Kronecker components depends on the layer input $\vx^{(i-1)}_n$.

In summary, we have the following dependencies of $\mA^{(i)}$ and $\mB^{(i)}$, which justifies their names:
\begin{align*}
  \mA^{(i)} &= \mA^{(i)}( \{\vx_{n}^{(i-1)}\}_n )
  \\
  \mB^{(i)} &= \mB^{(i)}( \{ (\jac_{\vx_n^{(i)}}\vf_{n})^{\top} \blacktriangle_{n,c}\}_{n,c})
\end{align*}

\paragraph{KFAC flavours and scaffold:}
There are many variations of KFAC and our plan is to cover as many as possible, while using the same code scaffold.
Some of them are motivated by computational efficiency.
Others are motivated by limits in which the approximation becomes exact.
The degrees of freedom are roughly as follows:
\begin{itemize}
\item For the input-based Kronecker factor, we will distinguish the presence of weight-sharing which will result in the so-called KFAC-expand and KFAC-reduce approximations.
  Weight sharing means that the same weight is used multiple times in the forward pass on a datum, e.g.\,when the layer is a convolution, or a linear layer that processes a list of vectors.

\item For the output gradient-based Kronecker factor, we will distinguish different choices to generate the backpropagated vectors $\blacktriangle_{n,c}$ which will allow us to tackle the approximation of different matrices (e.g.\,the Fisher versus empirical Fisher), or to reduce computational cost.
\end{itemize}

Then, the computation of KFAC looks as follows:
\begin{enumerate}
\item Perform a forward pass to compute $\{\vf_n\}_n $, storing the layer inputs $\{\vx_n^{(i-1)} \}_n$ and outputs $\{\vx_n^{(i)}\}_n$ of all layers $i$ for which a KFAC approximation is defined.

\item Compute the input-based Kronecker factors $\mA^{(i)}$ using the layer inputs $\{\vx_n^{(i-1)}\}_n$ taking into account the KFAC variation that is desired.

\item Generate the vectors $\{\blacktriangle_{n,c}\}_{n,c}$ to be backpropagated, taking into account the specified KFAC approximation. Compute the pseudo-gradients $\{(\jac_{\vx_n^{(i)}} \vf_n)^{\top} \blacktriangle_{n,c} \}_{n,c}$ w.r.t.\,the layer outputs. Finally, compute the output-based Kronecker factors $\mB^{(i)}$.

\item Account for scaling caused by the loss function's reduction $R$.

\item Return the KFAC approximation in the form $\mA^{(i)}, \mB^{(i)}$ for all supported $i$.
\end{enumerate}
This scaffold is implemented in \Cref{kfac}.
For the remaining part, we will now specify the details of how to compute the input-based and grad-output based Kronecker factors, as well as how to generate the backpropagated vectors, and how to test the resulting implementation for correctness.
%%% Local Variables:
%%% mode: latex
%%% TeX-master: "../main"
%%% End:
