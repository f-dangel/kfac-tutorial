There are many ways to flatten the entries of a tensor into a vector.
The two by far most common conventions are (i) last-varies-fastest ($\cvec$) and (ii) first-varies-fastest ($\rvec$).
Their names are easy to remember from their action on a matrix (see \Cref{ex:flattening}): $\cvec$-flattening concatenates columns into a vector (column flattening); $\rvec$-flattening concatenates rows into a vector (row flattening).
Column-flattening is popular in mathematical presentations, while row-flattening is popular in deep learning libraries which lay out tensors in row-major format in memory.
To see their differences, we will implement both.

For arbitrary tensors, we can generalize the matrix flattenings by ordering entries such that either their last index ($\cvec$, \Cref{def:cvec}) or first index ($\rvec$, \Cref{def:rvec}) varies fastest:

\switchcolumn[1]*
\codeblock{flattening}
\switchcolumn[0]

\begin{setup}\label{setup:flattening}
  Let $\tA \in \sR^{N_1 \times \dots \times N_A}$ be a tensor of rank $A$ whose entries are indexed through a tuple $(n_1, \dots, n_A)$ where $n_a \in \{1, \dots, N_a\}$.
\end{setup}
\begin{definition}[$\cvec$, \Cref{flattening}]\label{def:cvec}
  The first-varies-fastest flattening of $\tA$ from \Cref{setup:flattening} is
  \begin{align*}
    \cvec(\tA) =
    \begin{pmatrix}
      \etA_{1,1,\dots,1} \\
      \etA_{2,1,\dots,1} \\
      \vdots \\
      \etA_{N_1,1,\dots,1} \\
      \etA_{1,2,\dots,1} \\
      \vdots \\
      \etA_{N_1,2,\dots,1} \\
      \vdots \\
      \etA_{N_1,N_2,\dots,N_A}
    \end{pmatrix}
    \in \sR ^{N_1 \cdots N_A}\,.
  \end{align*}
\end{definition}

\begin{definition}[$\rvec$, \Cref{flattening}]\label{def:rvec}
  The last-varies-fastest flattening of $\tA$ from \Cref{setup:flattening} is
  \begin{align*}
    \rvec(\tA) =
    \begin{pmatrix}
      \etA_{1,\dots,1,1} \\
      \etA_{1,\dots,1,2} \\
      \vdots \\
      \etA_{1,\dots,1,N_A} \\
      \etA_{1,\dots,2,1} \\
      \vdots \\
      \etA_{1,\dots,2,N_A} \\
      \vdots \\
      \etA_{N_1,\dots,N_{A-1},N_A}
    \end{pmatrix}
    \in \sR ^{N_A \cdots N_1}\,.
  \end{align*}
\end{definition}

In code, we will sometimes require partial flattening of a sub-set of contiguous indices, instead of all indices.
The definitions are analogous, but the flattened indices are surrounded by static ones.

\begin{example}[Matrix flattening, \Cref{flattening}]\label{ex:flattening}
  For a matrix
  \begin{equation*}
    \mA = \begin{pmatrix} 1 & 2 \\ 3 & 4 \end{pmatrix}
  \end{equation*}
  we have
  \begin{equation*}
    \rvec(\mA)
    =
    \begin{pmatrix}
      1 \\ 2 \\ 3 \\ 4
    \end{pmatrix}\,,
    \qquad
    \cvec(\mA)
    =
    \begin{pmatrix}
      1 \\ 3 \\ 2 \\ 4
    \end{pmatrix}\,.
  \end{equation*}
\end{example}
%%% Local Variables:
%%% mode: latex
%%% TeX-master: "../main"
%%% End:
