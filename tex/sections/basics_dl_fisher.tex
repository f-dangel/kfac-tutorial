In Section ??
we have demonstrated that a neural network almost always models a likelihood for the labels given the inputs using parameters.
This perspective provides a new way to compare two models $f_1 \coloneq f(\bullet, \vtheta_1), f_2 \coloneq f(\bullet, \vtheta_{2})$ where $\vtheta_2 = \vtheta_1 + \Delta \vtheta$.
Instead of assessing their dissimilarity by their parameters' dissimilarity, e.g.\,using the Euclidean distance measure $d^2(f_1, f_2) = \left\lVert \vtheta_2 - \vtheta_1 \right\rVert_2^2 = \left\lVert \Delta \vtheta \right\rVert_2^2$, we could instead use the KL-divergence of the probability densities represented by the two networks, $d^2(f_1, f_2) = \mathrm{KL}(p(\rvx, \rvy \mid \vtheta_1) \mid\mid p(\rvx, \rvy \mid \vtheta_2))$.
% $\mathrm{KL}(r(\rvy \mid f(\vx, \vtheta_1)) \mid\mid r(\rvy \mid f(\vx, \vtheta_2)))$.
For general pairs $(\vtheta_1, \vtheta_2)$, the KL divergence is not a valid distance measure though, e.g.\,it is not symmetric w.r.t.\,its arguments.
However, for small $\Delta \vtheta$, the KL divergence can be locally approximated by a quadratic Taylor series.
The Hessian showing up in this approximation is the Fisher information matrix and it implies a proper metric inside a small neighbourhood,
\begin{align*}
  \mathrm{KL}(p(\rvx, \rvy \mid \vtheta) \mid\mid p(\rvx, \rvy \mid \vtheta + \Delta \vtheta))
  \\
  = -\frac{1}{2} {\Delta \vtheta}^{\top} \mF(\vtheta) \Delta \vtheta + \gO( (\Delta \vtheta)^3)\,.
\end{align*}
with the Fisher information matrix
\begin{align*}
  \mF(\vtheta) = \E_{p(\rvx, \rvy \mid \vtheta)} [-\hess_{\vtheta} \log p(\rvx, \rvy \mid \vtheta)]\,.
\end{align*}
Let's rewrite this expression in terms of the likelihood $p(\rvy \mid \vx, \vtheta)$, and its reparameterized likelihood $r(\rvy \mid f(\rvx, \vtheta))$,
\begin{align*}
  \mF(\vtheta)
  &=
    \E_{p_{\text{data}}(\rvx)}\E_{p(\rvy \mid \rvx, \vtheta)} [-\hess_{\vtheta} \log p(\rvy \mid \rvx, \vtheta)]\,.
  \\
  &=
    \E_{p_{\text{data}}(\rvx)}\E_{r(\rvy \mid f(\rvx, \vtheta))} [-\hess_{\vtheta} \log r(\rvy \mid f(\rvx, \vtheta))]\,.
\end{align*}
\paragraph{Type-II Fisher:} In practise, we must replace the true marginal distribution over the inputs with its empirical distribution $p_{\sD}(\rvx) = \frac{1}{N}\sum_n \delta(\rvx - \vx_n)$. We will call the resulting matrix the type-II Fisher information matrix (because it is defined via second-order derivatives of the likelihood) and use the abbreviation $\vf_n \coloneq f(\rvx = \vx_n, \vtheta)$,
\begin{align*}
  \mF^{\text{II}}(\vtheta)
  &=
    \E_{p_{\sD}(\rvx)}\E_{r(\rvy \mid f(\rvx, \vtheta))} [-\hess_{\vtheta} \log r(\rvy \mid f(\rvx, \vtheta))]\,.
  \\
  &=
    \frac{1}{N} \sum_n
    \E_{r(\rvy \mid \vf_n)} [-\hess_{\vtheta} \log r(\rvy \mid \vf_n)]\,.
\end{align*}
We can apply the chain rule to the Hessian of the log-likelihood and use the fact that $\E_{r(\rvy \mid \rvx, \vtheta)}[\nabla_{f(\rvx, \vtheta)} \log r(\rvy \mid f(\rvx, \vtheta))] = \vzero$ to simplify the type-II Fisher into
\begin{align*}
  \mF^{\text{II}}(\vtheta)
  \\
  &=
    \frac{1}{N} \sum_n
    \E_{r(\rvy \mid \vf_n)} [
    {\jac_{\vtheta} \vf_n}^{\top}
    \left( -\hess_{\vf_n} \log r(\rvy \mid \vf_n)  \right)
    \jac_{\vtheta} \vf_n
    ]\,.
    \shortintertext{(Jacobians do not depend on $\rvy$)}
  &=
    \frac{1}{N} \sum_n
    {\jac_{\vtheta} \vf_n}^{\top}
    \left(
    \E_{r(\rvy \mid \vf_n)} [
    -\hess_{\vf_n} \log r(\rvy \mid \vf_n)
    ]
    \right)
    \jac_{\vtheta} \vf_n
    \,.
\end{align*}

\paragraph{Type-I Fisher:} There is another way to rewrite the Fisher information matrix in terms of first-order derivatives of the log-likelihood.
We will call this the type-I Fisher information matrix.
To derive it, we use the property that the negative log-likelihood's Hessian equals its gradient covariance in expectation,
\begin{align*}
  \E_{p(\rvx, \rvy \mid \vtheta)}
  \left[
  -\hess_{\vtheta} \log p(\rvx, \rvy \mid \vtheta)
  \right]
  \\
  =
  \E_{p(\rvx, \rvy \mid \vtheta)}
  \left[
  -\nabla_{\vtheta} \log p(\rvx, \rvy \mid \vtheta)
  (-\nabla_{\vtheta} \log p(\rvx, \rvy\mid  \vtheta))^{\top}
  \right]\,.
\end{align*}
Hence, we can write the Fisher as
\begin{align*}
  \mF(\vtheta) &=
                 \E_{p(\rvx, \rvy \mid \vtheta)} [-\nabla_{\vtheta} \log p(\rvx, \rvy \mid \vtheta) (-\nabla_{\vtheta} \log p(\rvx, \rvy \mid \vtheta))^{\top}]
  \\
               &= \E_{p_{\text{data}}(\rvx)}\E_{p(\rvy \mid \rvx, \vtheta))}
                 [-\nabla_{\vtheta} \log p(\rvy \mid \rvx, \vtheta) (-\nabla_{\vtheta} \log p(\rvy \mid \rvx, \vtheta))^{\top}]
  \\
               &= \E_{p_{\text{data}}(\rvx)}\E_{r(\rvy \mid f(\rvx, \vtheta))}
                 [-\nabla_{\vtheta} \log r(\rvy \mid f(\rvx, \vtheta)) (-\nabla_{\vtheta} \log r(\rvy \mid f(\rvx, \vtheta)))^{\top}]
\end{align*}
Let's apply the chain rule for the gradient, substitute the empirical distribution for $p_{\text{data}}(\rvx)$ and simplify the notation using $\vf_n \coloneqq f(\rvx = \vx_n, \vtheta)$,
\begin{align*}
  \mF^{\text{I}}(\vtheta)
  \\
  &=
    \frac{1}{N} \sum_n
    \E_{r(\rvy \mid \vf_n)}
    [
    (-\nabla_{\vtheta} \log r(\rvy \mid \vf_n) (-\nabla_{\vtheta} \log r(\rvy \mid \vf_n))^{\top}
    ]
  \\
  &=
    \frac{1}{N} \sum_n
    \E_{r(\rvy \mid \vf_n)}
    [
    (\jac_{\vtheta}\vf_n)^{\top}
    (-\nabla_{\vf_n} \log r(\rvy \mid \vf_n) (-\nabla_{\vf_n} \log r(\rvy \mid \vf_n))^{\top}
    \jac_{\vtheta}\vf_n
    ]
  \\
  \shortintertext{(Jacobians do not depend on $\rvy$)}
  &=
    \frac{1}{N} \sum_n
    (\jac_{\vtheta}\vf_n)^{\top}
    \E_{r(\rvy \mid \vf_n)}
    [
    (-\nabla_{\vf_n} \log r(\rvy \mid \vf_n) (-\nabla_{\vf_n} \log r(\rvy \mid \vf_n))^{\top}
    ]
    \jac_{\vtheta}\vf_n \,.
\end{align*}

\paragraph{Monte-Carlo approximation of the Fisher:} In practise, we need to replace the expectation over $r(\rvy \mid \vf_n)$ with an estimator, e.g.\,a Monte-Carlo approximation with $M$ sampled labels $\tilde{\vy}_{n,1}, \dots, \tilde{\vy}_{n, M} \stackrel{\text{i.i.d}}{\sim} r(\rvy \mid \vf_n)$ for the prediction of datum $n$.

\begin{align*}
  \tilde{\mF}^{\text{I}}(\vtheta)
  \\
   =
    \frac{1}{NM} \sum_{n,m}
    (\jac_{\vtheta}\vf_n)^{\top}
      (-\nabla_{\vf_n} \log r(\rvy = \tilde{\vy}_{n,m} \mid \vf_n) (-\nabla_{\vf_n} \log r(\rvy = \tilde{\vy}_{n,m} \mid \vf_n))^{\top}
    \jac_{\vtheta}\vf_n
\end{align*}
Let's denote the $m$-th would-be gradient for sample $n$ by $\tilde{\vg}_{n,m} \coloneq -\nabla_{\vf_n} \log r(\rvy = \tilde{\vy}_{n,m} \mid \vf_n)$
We can then form the matrix
\begin{align*}
  \tilde{\mS}_n
  \coloneq
  \frac{1}{\sqrt{M}}
  \begin{pmatrix}
    \tilde{\vg}_{n,1} & \dots & \tilde{\vg}_{n,M}
  \end{pmatrix}
  \in \sR^{C \times M}
\end{align*}
for each $n$, and express the MC-approximated Fisher as
\begin{align*}
  \tilde{\mF}^{\text{I}}(\vtheta)
  &=
    \frac{1}{N} \sum_{n,m}
    (\jac_{\vtheta}\vf_n)^{\top}
    \tilde{\mS}_n
    \tilde{\mS}_n^{\top}
    \jac_{\vtheta}\vf_n
  \\
  &=
    \frac{1}{N} \sum_n
    \tilde{\mV}_n
    \tilde{\mV}_n^{\top}
\end{align*}
where $\tilde{\mV}_n \in \sR^{D \times M}$.
Did you notice what we just did here?
By stacking the would-be gradients into columns of a matrix, we were able to express the Fisher as self-outer product of a matrix $\tilde{\mV}_n$.
This notation looks very familiar from the GGN's self-outer product notation.
We can think of $\tilde{\mV}_n \in \sR^{D \times M}$ as a randomization of $\mV_n \in \sR^{D \times C}$ which can be much smaller (depending on the choice of $M$) and therefore cheaper to compute.

\paragraph{Reduction factor of the Fisher versus empirical risk:}
In the above derivation we arrived at an expression of the Fisher which uses a reduction factor $R = \frac{1}{N}$.
However, other reduction factors are equally common, e.g.\,when predicting a sequence of $T$ tokens, the reduction factor is $R = \frac{1}{NT}$.
Note though that had we considered the statistical manifold over likelihoods $\{ p(\rvy \mid \rvx, \vtheta) \mid \vtheta \in \Theta \}$ rather than that over joint distributions $\{ p(\rvx, \rvy \mid \vtheta) \mid \vtheta \in \Theta \}$, our derivation of the Fisher would lack the reduction of $\frac{1}{N}$.
In other words, the Fisher is tied to a statistical manifold over probability densities, and depending on our choice of considering likelihood or joint distributions, we obtain different normalizations.
This is a subtlety that we will usually ignore in practise, simply by using the reduction factor from our empirical risk (which consequently implies the statistical manifold on which we are considering the Fisher).

TODO Could create a table here were we map the different loss functions and reductions to the corresponding statistical manifolds.

TODO Write code to draw sampled gradients.
Compare it with the (sampled) Hessians.

\begin{example}[Gradients of MSELoss]
  Consider the square loss from \Cref{ex:square_loss} and its probabilistic interpretation from \Cref{ex:square_loss_probabilistic}
  For some label $\vy$, the gradient is given by
  \begin{align*}
    \nabla_{\vf} c(\vf, \vy)
    &=
      - \nabla_{\vf} \log r(\rvy =\vy \mid \vf)
    \\
    &=
      \vf - \vy\,.
  \end{align*}
\end{example}

\begin{example}[Gradients of CrossEntropyLoss]
  Consider the softmax cross-entropy loss from \Cref{ex:cross_entropy_loss} and its probabilistic interpretation from \Cref{ex:cross_entropy_loss_probabilistic}. For some label $y$, the gradient is given by
  \begin{align*}
    \nabla_{\vf} c(\vf, y)
    &=
      - \nabla_{\vf} \log r(\ry = y \mid \vf)
    \\
    &=
      p(\vf) - \mathrm{onehot}(y)\,.
  \end{align*}
\end{example}

TODO Figure on the right showing the type-I Fisher and the type-II Fisher on 1 versus 100 samples.
If the reader is curious why the type-II Fisher has no variation at all, point them to the next section on the connection between GGN and Fisher.
%%% Local Variables:
%%% mode: latex
%%% TeX-master: "../main"
%%% End:
