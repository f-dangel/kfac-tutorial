\switchcolumn[1]*
\codeblock{basics/reduction_factors}
\switchcolumn[0]

This tutorial is meant to be self-contained.
Therefore, we will start with an extensive introduction to KFAC-relevant concepts.
This allows us to build the code functionality we will later need to verify our implementation.

\paragraph{Roadmap.} First, we introduce the empirical risk (`the loss', \cref{subsec:empirical-risk-minimization}) whose curvature KFAC approximates, and neural networks (\cref{subsec:deep-neural-networks}).
One recurring theme in our discussion will be that the loss and neural net have probabilistic interpretations in most deep learning settings: Minimizing the empirical risk corresponds to maximum likelihood estimation where the neural net models a likelihood (\cref{subsec:probabilistic-interpretation}).
Next, since curvature information is based on the Hessian which contains second-order partial derivatives, we will talk about first- and second-order derivatives, and how to compute with them using PyTorch's automatic differentiation (\cref{subsec:derivatives}).
We conclude with an introduction to all curvature matrices relevant to our discussion (\cref{subsec:curvature-matrices}).
These include the Hessian, generalized Gauss-Newton (GGN) matrix and different flavours of the Fisher information matrix which follows from the probabilistic interpretation from \cref{subsec:probabilistic-interpretation}.

\subsection{Empirical Risk Minimization}\label{subsec:empirical-risk-minimization}
We consider supervised learning with a neural network $f: \gX \times \Theta \to \gF$ that maps a given input $\vx \in \gX$ from an input domain $\gX$ to a prediction $f(\vx, \vtheta) \in \gF$ in a prediction domain $\gF$ using parameters $\vtheta \in \Theta$ from a parameter space $\Theta$.
Predictions are scored with a criterion function $c: \gF \times \gY \to \sR$ that compares the prediction to the true target $\vy \in \gY$ from a label space $\gY$, producing a single number called the loss on datum $(\vx, \vy)$.
For a data set $\sD = \{(\vx_n, \vy_n) \mid n=1, \dots, N\}$ of collected labelled examples, we evaluate the per-datum criteria and accumulate them into the total loss, using an accumulation factor $R \in \sR$ (common choices are $\nicefrac{1}{N}, 1, \nicefrac{1}{N \dim(\gY)}$),
\begin{align}\label{eq:empirical_risk}
  \begin{split}
    \gL(\vtheta; \sD) & = R \sum_{n=1}^N \ell_n(\vtheta)
    \\
                      & = R \sum_{n=1}^N c(f(\vx_n, \vtheta), \vy_n)\,.
  \end{split}
\end{align}
One goal of training is to find the parameters $\vtheta$ that reduce the empirical risk $\gL(\vtheta; \sD)$, while avoiding overfitting to the training data.

This abstraction into neural network, criterion, and reduction will become useful later.
Below, we relate it to the PyTorch implementations of loss functions used for regression and classification.

\begin{example}[Square loss, \Cref{reduction_factors}]\label{ex:square_loss}
  For least squares regression, with vector-valued targets ($\gY = \sR^C = \gF$), PyTorch's \texttt{torch.nn.MSELoss} uses the following criterion function and reduction constant:
  \begin{align*}
    c(\vf, \vy)
    =
    \frac{1}{2}\sum_{c=1}^C [\vf - \vy]_c^2
  \end{align*}
  and
  \begin{align*}
    R
    =
    \begin{cases}
      2                     & \text{\texttt{reduction="sum"}}
      \\
      \frac{2}{N \dim(\gY)} & \text{\texttt{reduction="mean"}}
    \end{cases}
  \end{align*}
  where $\dim(\gY) = C$ in the vector case, but $\gY = \gF$ could also be matrix or tensor spaces in a more general setup.
\end{example}

\begin{example}[Cross-entropy loss, \Cref{reduction_factors}]\label{ex:cross_entropy_loss}
  For classification, with categorical targets ($\gY = \{1, \dots, C\}$ and $\gF = \sR^C$), PyTorch's \texttt{torch.nn.CrossEntropyLoss} uses the following criterion function and reduction constant:
  \begin{align*}
    c(\vf, y)
    =
    - \log([\softmax(\vf)]_y)
  \end{align*}
  and
  \begin{align*}
    R
    =
    \begin{cases}
      1                     & \text{\texttt{reduction="sum"}}
      \\
      \frac{1}{N \dim(\gY)} & \text{\texttt{reduction="mean"}}
    \end{cases}
  \end{align*}
  with $\softmax(\vf)_i = \nicefrac{\exp([\vf]_i)}{\sum_{j=1}^C \exp([\vf]_{j})}$.
  For the vector case $\dim(\gY) = 1$ but $\gY, \gF$ could also be compatible matrix or tensor spaces in a more general setup where we aim to classify sequences of categorical labels.
\end{example}
%%% Local Variables:
%%% mode: latex
%%% TeX-master: "../main"
%%% End:


\switchcolumn[1]
\codeblock{basics/forward_pass}
\switchcolumn[0]

\subsection{Deep Neural Networks}\label{subsec:deep-neural-networks}
For simplicity, we consider sequential neural networks $f: \gX \times \Theta \to \gF$ that map a given input $\vx$ to a prediction $f(\vx, \vtheta)$ using parameters $\vtheta$.
$\gX$ is the input space, $\Theta$ is the parameter space, and $\gF$ is the prediction space.

The network is composed of $L$ layers $f^{(i)}(\cdot, \vtheta^{(i)}), i=1,\dots, L$, each of which can have its own parameters $\vtheta^{(i)}$.
The whole network is simply a stack of layers, i.e.
$f = f^{(L)} \circ \dots \circ f^{(1)}$, and the its evaluation produces a set of intermediate features,
\begin{align*}
  \vx^{(i)} = f^{(i)}(\vx^{(i-1)}, \vtheta^{(i)}), \quad i=1,\dots, L,
\end{align*}
starting with $\vx^{(0)} \leftarrow \vx$, ending in $\vx^{(L)} \leftarrow f(\vx, \vtheta)$.
We refer to the hidden representations $\vx^{(i-1)}$ as the \emph{input to layer $i$}, and to $\vx^{(i)}$ as the \emph{output of layer $i$}.
For some layers, the parameters will be empty, e.g.\,activation, pooling, or dropout layers.
To compute KFAC, we will need to access to the inputs and outputs of certain layers.
We can obtain these by intercepting the forward pass of a neural network using PyTorch's hook mechanism, see \Cref{forward_pass}.

%%% Local Variables:
%%% mode: latex
%%% TeX-master: "../main"
%%% End:


\subsection{Probabilistic Interpretation}\label{subsec:probabilistic-interpretation}
\begin{caveat}
  Implementations of loss functions mix the concepts of criterion and reduction.
  This is often fine, but sometimes makes it difficult to map between these components without accidentally forgetting a factor.
  We need to keep both concepts separate to reduce the chance of introducing scaling bugs.
\end{caveat}

So far, we have only considered minimizing a risk over a data set given an arbitrary criterion function $c$.
Here, we want to take one step back and first describe risk minimization from a probabilistic perspective.
After, we will derive a very similar form but using a probabilistic argument via maximum likelihood estimation.

Recall the definition of the risk in \cref{eq:empirical_risk}.
We want to minimize the risk over the data set $\sD$,
\begin{align*}
  \min_{\vtheta} \gL(\vtheta; \sD) = \min_{\vtheta} R \sum_{n=1}^N \ell_n(\vtheta)\,.
\end{align*}
Let's express this in terms of an expectation.
To do so, assume that there exists a data-generating process $p_{\text{data}}(\rvx, \rvy)$ over input-target pairs.
Ideally, we would like to minimize the expected risk over that distribution, i.e.\,$\argmin_{\vtheta} \E_{(\vx, \vy) \sim p_{\text{data}}(\rvx, \rvy)}[c(f(\vx, \vtheta), \vy)]$.
However, $p_{\text{data}}$ is not tractable.
Therefore, we draw a finite amount of samples and collect them into a data set
\begin{align*}
  \sD = \{ (\vx_n, \vy_n) \mid (\vx_n, \vy_n) \stackrel{\text{i.i.d.}}{\sim} p_{\text{data}}(\vx, \vy) \}\,.
\end{align*}
Then, we can replace the intractable data-generating process $p_{\text{data}}$ with the empirical distribution that consists of a uniformly weighted sum of delta peaks around the collected data points,
\begin{align*}
  p_{\sD}(\rvx, \rvy) = \frac{1}{N} \sum_{n=1}^N \delta(\rvx - \vx_n) \delta(\rvy - \vy_n)\,.
\end{align*}
This turns risk minimization into a tractable task,
\begin{align*}
  \argmin_{\vtheta} \E_{(\vx, \vy) \sim p_{\text{data}}(\rvx, \rvy)}[c(f(\vx, \vtheta), \vy)]
  \\
  \approx
  \\
  \argmin_{\vtheta} \E_{(\vx, \vy) \sim p_{\sD}(\rvx, \rvy)}[c(f(\vx, \vtheta), \vy)]
  \\
  =
  \\
  \argmin_{\vtheta} \frac{1}{N} \sum_n c(f(\vx_n, \vtheta), \vy_n)
  \\
  \Leftrightarrow
  \\
  \argmin_{\vtheta} \gL(\vtheta; \sD)\,.
\end{align*}
Note that in the last step we used that we can arbitrarily an objective without changing the location of its minima to match the scaling factor with the reduction factor.
So now we have shown that empirical risk minimization can be written as minimizing an objective that is a (scaled) expectation value over an empirical density $p_{\sD}(\rvx, \rvy)$.

Let's approach this from a probabilistic perspective now.
Assume we want to learn $p_{\text{data}}(\rvx, \rvy) = p_{\text{data}}(\rvy \mid \rvx) p_{\text{data}}(\vx)$ using a parameterized density of the form $p(\rvx, \rvy \mid \vtheta) = p(\rvy \mid \rvx, \vtheta) p_{\text{data}}(\rvx)$ where $p_{\text{data}}(\rvx) = \int p_{\text{data}}(\rvx, \rvy) \mathrm{d}\rvy$ is the marginal density of the input data.
Note that we model only the likelihood for the labels with parameters $\vtheta$.
One plausible approach to make $p$ resemble $p_{\text{data}}$ is to minimize their KL divergence,
\begin{align*}
  \argmin_{\vtheta} \mathrm{KL}(p_{\text{data}}(\rvx, \rvy) \mid\mid p(\rvx, \rvy \mid \vtheta))\,.
\end{align*}
We can simplify this expression by substituting the definition of the KL divergence and dropping terms that do not depend on $\vtheta$,
\begin{align*}
  \Leftrightarrow
  \\
  \argmin_{\vtheta} \E_{p_{\text{data}}(\rvx, \rvy)}[- \log( p(\rvx, \rvy \mid \vtheta))]\,.
\end{align*}
Note that this looks very similar to the expected risk from above.
Next, let's factorize our model distribution using its conditional and marginal densities and drop terms that do not depend on $\vtheta$, which yields
\begin{align*}
  \Leftrightarrow
  \\
  \argmin_{\vtheta} \E_{p_{\text{data}}(\rvx, \rvy)}[- \log( p(\rvy \mid \vtheta))]\,.
\end{align*}
To make this problem tractable, we need to replace the intractable data-generating process $p_{\text{data}}$ with the empirical distribution $p_{\sD}$:
\begin{align*}
  \approx
  \\
  \argmin_{\vtheta} \E_{p_{\sD}(\rvx, \rvy)}[- \log( p(\rvy \mid \vtheta))]
  \\
  =
  \\
  \argmin_{\vtheta} \frac{1}{N} \sum_n - \log( p(\rvy = \vy_n \mid \rvx=\vx_n,\vtheta))
  \\
  =
  \\
  \argmin_{\vtheta} R \sum_n - \log( r(\rvy = \vy_n \mid f(\vx_n,\vtheta)))
\end{align*}
In the last step we have made two changes to make the expression resemble more that of the empirical risk: first, we introduced a new scaling; second, we changed the likelihood's parameterization from $p(\rvy \mid \rvx, \vtheta)$ to $r(\rvy \mid f(\rvx, \vtheta))$ where $f$ is a neural net. This parameterization makes it clear that the neural network represents a conditional distribution over the labels given the inputs (and parameters).

We are now very close to writing down the explicit connection between empirical risk minimization and maximum likelihood estimation.
The last remaining step is to connect the model's likelihood $r(\rvy \mid \rvx, \vtheta)$ with the criterion function $c(f(\vx, \vtheta), \vy)$ from empirical risk minimization.
The following two examples show that empirical risk minimization with square or softmax cross-entropy loss can be interpreted as maximum likelihood (or equivalently, minimum negative log-likelihood) estimation problem where the neural network parameterizes a Gaussian and a categorical distribution over the labels:

\begin{example}[Probabilistic interpretation of square loss]\label{ex:square_loss_probabilistic}
  For square loss from \Cref{ex:square_loss}, we have that $c(\vf, \vy) = - \log( \mathrm{const.}
  \cdot \gN(\rvy \mid \vmu = f(\vx, \vtheta), \mSigma = \mI))$ where $\gN(\bullet | \vmu, \mSigma)$ is a multi-variate Gaussian distribution with mean $\vmu \in \sR^C$ and positive definite covariance $\mSigma \in \sR^{C \times C}$,
  \begin{align*}
    \gN(\rvy \mid \vmu, \mSigma)
    =
    \frac{1}{{(2\pi)}^{C/2} \sqrt{\det(\mSigma)}}
    \\
    \exp\left( -\frac{1}{2} {(\rvy - \vmu)}^\top \mSigma^{-1} (\rvy - \vmu) \right)\,.
  \end{align*}
  We can safely neglect the constant factor for the optimization problem and therefore identify
  \begin{align*}
    c = \text{\texttt{MSELoss}}
    \\
    \Leftrightarrow
    \\
    r(\rvy \mid f(\rvx, \vtheta)) &= \gN(\rvy \mid \vmu = f(\vx, \vtheta), \mSigma = \mI)\,
    \\
    p(\rvy \mid \rvx, \vtheta) &= \gN(\rvy \mid \vmu = f(\vx, \vtheta), \mSigma = \mI)\,.
  \end{align*}
\end{example}

\begin{example}[Probabilistic interpretation of cross-entropy loss]\label{ex:cross_entropy_loss_probabilistic}
  For cross-entropy loss from \Cref{ex:cross_entropy_loss}, we have that $c(\vf, y) = - \log( \gC(\ry \mid \vsigma = \softmax(\vf) ))$ where $\gC(\bullet | \vsigma)$ is a categorical distribution over $\{1, \dots, C\}$ with probabilities $\vsigma \in \sR^C_{\ge 0}$ and $\vsigma^\top \vone = 1$,
  \begin{align*}
    \gC(\ry \mid \vsigma)
    =
    \prod_{c=1}^C [\vsigma]_c^{\delta_{\ry,c}}\,.
  \end{align*}
  Hence, we can identify
  \begin{align*}
    r(\ry \mid f(\rvx, \vtheta)) &= \gC(\ry \mid \vsigma = \softmax(f(\vx, \vtheta)))\,
    \\
    p(\ry \mid \rvx, \vtheta) &= \gC(\ry \mid \vsigma = \softmax(f(\vx, \vtheta)))\,.
  \end{align*}
\end{example}
%%% Local Variables:
%%% mode: latex
%%% TeX-master: "../main"
%%% End:


\subsection{Derivatives \& Automatic Differentiation}\label{subsec:derivatives}

\begin{caveat}
  In deep learning, we often work with matrices, or higher-dimensional tensors.
  We want to use matrix linear algebra expressions to avoid using heavy index notation.
  This can be achieved by flattening all tensors back into vectors and re-using definitions of derivatives from the vector case.
  However, we must be careful when translating the results back to the tensor format, as the translation process depends on the flattening convention.
  Classically, the mathematical derivations prefer a \emph{different} flattening scheme than the one used in deep learning libraries.
\end{caveat}

\switchcolumn[0]*
\subsubsection{Flattening}
\switchcolumn[1]
\begin{example}[Matrix flattening, \Cref{basics/flattening}]\label{ex:flattening}
  For a matrix
  \begin{equation*}
    \mA = \begin{pmatrix} 1 & 2 \\ 3 & 4 \end{pmatrix}
  \end{equation*}
  we have
  \begin{equation*}
    \rvec(\mA)
    =
    \begin{pmatrix}
      1 \\ 2 \\ 3 \\ 4
    \end{pmatrix}\,,
    \qquad
    \cvec(\mA)
    =
    \begin{pmatrix}
      1 \\ 3 \\ 2 \\ 4
    \end{pmatrix}\,.
  \end{equation*}
\end{example}
\switchcolumn[0]

\vspace{\baselineskip}
\begin{caveat}[Flattening]
  In deep learning, we often work with matrices, or higher-dimensional tensors.
  We want to use matrix linear algebra expressions to avoid using heavy index notation.
  This can be achieved by flattening all tensors back into vectors and reusing definitions of derivatives from the vector case.
  However, we must be careful when translating the results back to the tensor format, as the translation process depends on the flattening convention.
  Classically, the mathematical derivations prefer a \emph{different} flattening scheme than the one used in deep learning libraries.
  This can cause confusion and bugs.
\end{caveat}

\switchcolumn[1]
\codeblock{basics/flattening}
\switchcolumn[0]

There are many ways to flatten the entries of a tensor into a vector.
The two by far most common conventions are (i) last-varies-fastest ($\rvec$) and (ii) first-varies-fastest ($\cvec$).
Their names are easy to remember from their action on a matrix (see \Cref{ex:flattening}): $\cvec$-flattening concatenates columns into a vector (column flattening); $\rvec$-flattening concatenates rows into a vector (row flattening).

Column-flattening is popular in mathematical presentations, while row-flattening is popular in deep learning libraries, which lay out tensors in row-major format in memory.
To see their differences, we will implement both (\Cref{basics/flattening}).
For arbitrary tensors, we can generalize the matrix flattenings by ordering entries such that either their first index ($\cvec$, \Cref{def:cvec}) or last index ($\rvec$, \Cref{def:rvec}) varies fastest:


\begin{setup}[Rank-$A$ tensor]\label{setup:flattening}
  Let $\tA \in \sR^{N_1 \times \dots \times N_A}$ be a tensor of rank $A$ whose entries are indexed through a tuple $(n_1, \dots, n_A)$ where $n_a \in \{1, \dots, N_a\}$ for $a \in \{1, \dots, A\}$.
  Vectors are rank-1 tensors, and matrices are rank-2 tensors.
\end{setup}
\begin{definition}[$\cvec$, \Cref{basics/flattening}]\label{def:cvec}
  The first-varies-fastest flattening of tensor $\tA$ from \Cref{setup:flattening} is
  \begin{align*}
    \cvec(\tA) =
    \begin{pmatrix}
      \etA_{\colored{1},1,\dots,1}   \\
      \etA_{\colored{2},1,\dots,1}   \\
      \vdots               \\
      \etA_{\colored{N_1},1,\dots,1} \\
      \etA_{\colored[VectorPink]{1},2,\dots,1}   \\
      \vdots               \\
      \etA_{\colored[VectorPink]{N_1},2,\dots,1} \\
      \vdots               \\
      \etA_{N_1,N_2,\dots,N_A}
    \end{pmatrix}
    \in \sR ^{N_1 \cdots N_A}\,.
  \end{align*}
\end{definition}

\begin{definition}[$\rvec$, \Cref{basics/flattening}]\label{def:rvec}
  The last-varies-fastest flattening of tensor $\tA$ from \Cref{setup:flattening} is
  \begin{align*}
    \rvec(\tA) =
    \begin{pmatrix}
      \etA_{1,\dots,1,\colored{1}}   \\
      \etA_{1,\dots,1,\colored{2}}   \\
      \vdots               \\
      \etA_{1,\dots,1,\colored{N_A}} \\
      \etA_{1,\dots,2,\colored[VectorPink]{1}}   \\
      \vdots               \\
      \etA_{1,\dots,2,\colored[VectorPink]{N_A}} \\
      \vdots               \\
      \etA_{N_1,\dots,N_{A-1},N_A}
    \end{pmatrix}
    \in \sR ^{N_A \cdots N_1}\,.
  \end{align*}
\end{definition}

In code, we will sometimes require partial flattening of a subset of contiguous indices, instead of all indices (\eg to turn a tensor into a matrix by first flattening the row indices, followed by flattening the column indices).
The definitions are analogous, but the flattened indices are surrounded by static ones.
%%% Local Variables:
%%% mode: latex
%%% TeX-master: "../main"
%%% End:


\switchcolumn[0]*
\subsubsection{Jacobians, JVP, VJPs}
Building up to curvature approximations that tackle the approximation of second-order partial derivatives, we start with first-order derivatives.
These are collected into a matrix called the Jacobian, which depends on the flattening convention.
We can multiply with the Jacobian and its transpose via automatic differentiation, without building up the matrix in memory.
These operations are called vector-Jacobian products (VJPs) and Jacobian-vector products (JVPs), respectively.
Machine learning libraries like JAX and PyTorch offer routines for computing Jacobians, VJPs, and JVPs.
However, their interface is functional.
Here, we provide an alternative implementation which accepts nodes of a computation graph rather than functions as input and will be beneficial for modular implementations of neural networks, as we consider later.
We also provide examples for important Jacobians, namely the output-parameter Jacobian of an affine map, i.e.\, a linear layer.
These Jacobians exhibit Kronecker structure, which is the foundation for the `K' in KFAC.
We also verify this structure numerically.
Crucially, the Kronecker structure changes depending on the flattening convention.

\begin{setup}[Vector-to-vector function]\label{setup:vector_to_vector_function}
  Let function $f: \sR^A \to \sR^B, \va \mapsto \vb = f(\va)$ denote a vector-to-vector function.
\end{setup}

\begin{definition}[Jacobian of a vector-to-vector function]\label{def:vector_jacobian}
  The Jacobian of a vector-to-vector function $f$ from \Cref{setup:vector_to_vector_function}, $\jac_{\va}\vb \in \sR^{B \times A}$, collects the first-order partial derivatives into a matrix such that
  \begin{align*}
    [\jac_{\va} \vb]_{i,j} = \frac{\partial [f(\va)]_i}{\partial [\va]_j}\,.
  \end{align*}
\end{definition}
\Cref{def:vector_jacobian} is limited to vector-to-vector functions.
The more general Jacobian of a tensor-to-tensor function can be indexed with combined indices from the input and output domain:

\begin{setup}[Tensor-to-tensor function]\label{setup:jacobians}
  Consider a tensor-to-tensor function $f: \sR^{A_1 \times \dots \times A_N} \to \sR^{B_1 \times \dots \times B_M}, \tA \mapsto \tB = f(\tA)$ from a rank-$N$ tensor $\tA$ into a rank-$M$ tensor $\tB$.
\end{setup}

\begin{definition}[General Jacobian, \Cref{jacobians}]\label{def:general_jacobian}
  The general Jacobian of $f$ from \Cref{setup:jacobians}, $\tJ_{\tB}\tA$, is a rank-$(M+N)$ tensor that collects the first-order partial derivatives such that
  \begin{align*}
    [\tJ_{\tA}\tB]_{j_1, \cdots ,j_M, i_1, \cdots, i_N} = \frac{\partial [f(\tA)]_{j_1, \cdots, j_M}}{\partial [\tA]_{i_1, \cdots, i_N}}\,.
  \end{align*}
\end{definition}
For $M=N=1$, the general Jacobian reduces to the Jacobian of a vector-to-vector function from \Cref{def:vector_jacobian}.

\paragraph{Multiplication} In practise, this general Jacobian can be prohibitively large and therefore one must almost always work with it in a matrix-free fashion, i.e.\, through VJPs and JVPs.

\switchcolumn[1]*
\codeblock{basics/jacobian_products}
\switchcolumn[0]

\begin{definition}[Vector-Jacobian products (VJPs), \Cref{jacobian_products}]\label{def:vjp}
  Given a tensor-to-tensor function $f$ from \Cref{setup:jacobians} and a tensor $\tV \in \sR^{B_1 \times \dots \times B_M}$ in the output domain, the vector-Jacobian product (VJP) $\tU$ of $\tV$ and $\tJ_{\tB}\tA$ lives in the $f$'s input domain and follows by contracting out the output indices,
  \begin{align*}
    &[\tU]_{i_1, \dots, i_N}
    \\
    &=
      \sum_{j_1, \dots, j_M}
      [\tV]_{j_1, \dots, j_M}
      [\tJ_{\tA}\tB]_{j_1, \dots, j_M, i_1, \dots, i_N}\,.
  \end{align*}
\end{definition}
For $M=N=1$, $\tV, \tU \to \vv, \vu$ are column vectors, $\tJ_{\tA}\tB \to \jac_{\va}\vb$ is a matrix, and the VJP is $\vu^{\top} = \vv^{\top} (\jac_{\va}\vb)$ or $\vu = (\jac_{\va}\vb)^{\top} \vv$, i.e.\,multiplication with the transpose Jacobian.

VJPs are at the heart of reverse-mode automatic differentiation, aka backpropagation (this is why $\tU$ is often called the \emph{pull-back} of $\tV$ through $f$).
Therefore, they are easy to implement with standard functionality.

The other popular contraction is between the Jacobian and a vector from the function's input domain that yields:

\begin{definition}[Jacobian-vector products (JVPs), \Cref{jacobian_products}]\label{def:jvp}
  Given a tensor-to-tensor function $f$ from \Cref{setup:jacobians} and a tensor $\tV \in \sR^{A_1 \times A_N}$ in the input domain, the Jacobian-vector product (JVP) $\tU$ between $\tV$ and $\tJ_{\tB}\tA$ lives in $f$'s output domain and follows by contracting the output indices,
  \begin{align*}
    &[\tU]_{j_1, \dots, j_M}
    \\
    &=
      \sum_{i_1, \dots, i_N}
      [\tJ_{\tA}\tB]_{j_1, \dots, j_M, i_1, \dots, i_N}
      [\tV]_{i_1, \dots, i_N}\,.
  \end{align*}
\end{definition}
For the vector case, $\tU, \tV, \tJ_{\tA}\tB \to \vu, \vv, \jac_{\va}\vb$, the JVP is $\vu = (\jac_{\va}\vb) \vv$, as suggested by its name.
JVPs are common in forward mode automatic differentiation ($\tU$ is often called the \emph{push-forward} of $\tV$ through $f$).
Only recently has this mode garnered attention.
Hence, the current JVP functionality in ML libraries usually follows a functional API.
To obtain an implementation that accepts variables from a computation graph and is more compatible with the modular approach we chose in this tutorial, we can use a trick\footnote{See \url{https://j-towns.github.io/2017/06/12/A-new-trick.html}.} that implements a VJP using two VJPs.

Jacobian products are efficient, but somewhat more abstract to work with as we cannot `touch' the full tensor. Also, we often also would like to think about this tensor as a matrix to be able to present derivations using linear algebra notation.

\switchcolumn[1]*
\codeblock{basics/jacobians}
\switchcolumn[0]

\paragraph{Matricization} We can reduce the general Jacobian back to the Jacobian matrix in two different ways: We can either directly matricize the tensor, or `flatten' the function $f \to f^{\vec}$ such that it consumes and produces vectors, then compute its Jacobian.
Both ways and their resulting Jacobian matrices depend on the flattening convention we choose.
The following definitions are consistent in the sense that both of the aforementioned approaches yield the same result, illustrated by this commutative diagram:

\begin{figure}[!h]
  \centering
  \begin{tikzpicture}[%
    font=\scriptsize,%
    thick,
    box/.style = {rectangle, draw=black, rounded corners, fill=VectorGray!50},%%
    ]
    \node[box] (A) at (0,0) {$f: \tA \mapsto \tB = f(\tA)$};
    \node[box] (B) at (4,0) {$\jac_{\tA}\tB$};
    \node[box, align=center] (C) at (0,-2.5) {%
      $f^{\vec} = \vec \circ f \circ \vec^{-1}$\\%
      $\vec(\tA) \coloneq \va \mapsto \vec(\tB) \coloneq \vb$%
    };
    \node[box, align=center] (D) at (4,-2.5) {%
      $\jac_{\va} \vb$\\%
      $=$\\%
      $\mat(\tJ_{\tA}\tB)$\\%
      $\coloneq$\\%
      $\jac^{\vec}_{\tA}\tB$%
    };
    \draw[-Stealth] (A.east) -- node[fill=white] {$\tJ$} (B.west);
    \draw[-Stealth] (A.south) -- node[fill=white] {flatten $f$} (C.north);
    \draw[-Stealth] (C.east) -- node[fill=white] {$\jac$} (D.west);
    \draw[-Stealth] (B.south) -- node[fill=white] {matricize $\jac_{\tA}\tB$} (D.north);
  \end{tikzpicture}
  \caption{Flattening and taking the Jacobian commute and lead to the same matricized Jacobian.
    $\vec$ denotes one of the flattening conventions from \Cref{def:cvec,def:rvec}.
    $\mat$ denotes matricization and involves two flattenings for the row and column dimensions, respectively.}
\end{figure}

Hence, there are two of interest for the purpose of this tutorial: the $\cvec$- and $\rvec$-Jacobians.
The $\cvec$-Jacobian is used in mathematical derivations in the literature.
The $\rvec$-Jacobian is common in code.

\begin{definition}[$\cvec$-Jacobian, \Cref{jacobians}]\label{def:cvec_jacobian}
  For a tensor-to-tensor function $f$ from \Cref{setup:jacobians}, its $\cvec$-Jacobian $\jac^{\cvec}_{\tA}\tB \in \sR^{A_N \cdots A_1 \times B_M \cdots B_1}$ results from flattening input and output tensors with $\cvec$ and applying the Jacobian definition for vectors,
  \begin{align*}
    [\jac^{\cvec}_{\tA}\tB]_{i,j}
    =
    \frac{\partial [\cvec(f(\tA))]_i}{\partial [\cvec(\tA)]_j}\,.
  \end{align*}
\end{definition}

\begin{definition}[$\rvec$-Jacobian, \Cref{jacobians}]\label{def:rvec_jacobian}
  For a tensor-to-tensor function $f$ from \Cref{setup:jacobians}, its $\rvec$-Jacobian $\jac^{\rvec}_{\tA}\tB \in \sR^{A_1 \cdots A_N \times B_1 \cdots B_M}$ results from flattening input and output tensors with $\rvec$ and applying the Jacobian definition for vectors,
  \begin{align*}
    [\jac^{\rvec}_{\tA}\tB]_{i,j}
    =
    \frac{\partial [\rvec(f(\tA))]_i}{\partial [\rvec(\tA)]_j}\,.
  \end{align*}
\end{definition}

As we will see in the following examples, the two Jacobians usually differ from each other, albeit in subtle ways. We highlight their differences on a linear layer, which will be useful later on when we discuss KFAC.

\switchcolumn[1]*
\codeblock{basics/jacobians_linear_layer}
\switchcolumn[0]

\begin{example}[$\cvec$- and $\rvec$-weight Jacobians of a linear layer, \Cref{jacobians_linear_layer}]\label{ex:weight_jacobians_linear_layer}
  Consider an affine map with weight matrix $\mW \in \sR^{D_{\text{out}} \times D_{\text{in}}}$, bias vector $\vb \in \sR^{D_{\text{out}}}$, input vector $\vx \in \sR^{D_{\text{in}}}$ and output vector $\vz \in \sR^{D_{\text{out}}}$ with
  \begin{align*}
    \vz
    \coloneqq
    \mW \vx + \vb
    =
    \begin{pmatrix}
      \mW & \vb
    \end{pmatrix}
    \begin{pmatrix}
      \vx \\ 1
    \end{pmatrix}
    \coloneqq
    \tilde{\mW}
    \tilde{\vx}\,.
  \end{align*}
  To express this operation as matrix-vector multiplication, we combined weight and bias into a single matrix $\tilde{\mW}$ and augment the input with a one, yielding $\tilde{\vx}$, to account for the bias contribution.

  The $\cvec$-Jacobian w.r.t.\,the combined weight is
  \begin{align*}
    \jac^{\cvec}_{\tilde{\mW}}\vz
    =
    \tilde{\vx}^{\top}
    \otimes
    \mI_{D_{\text{out}}}\,.
  \end{align*}
  In contrast, the $\rvec$-Jacobian is
  \begin{align*}
    \jac^{\cvec}_{\tilde{\mW}}\vz
    =
    \mI_{D_{\text{out}}}
    \otimes
    \tilde{\vx}^{\top}\,,
  \end{align*}
  see \Cref{jacobians_linear_layer}.
  Note that the order of Kronecker factors is \emph{reversed}, depending on the flattening scheme.
\end{example}

\begin{example}[$\cvec$- and $\rvec$-weight Jacobians of a linear layer with weight sharing]
  Consider the same affine map from above, but now processing multiple input vectors $\mX = \begin{pmatrix}\vx_1 & \dots & \vx_S\end{pmatrix} \in \sR^{D_{\text{in}}\times S}$, yielding a sequence $\mZ = \begin{pmatrix} \vz_1 & \dots & \vz_S\end{pmatrix} \in \sR^{D_{\text{out}}\times S}$ where each $\vz_s$ is produced like above.
  The parameters are \emph{shared} over all vectors in the input sequence.
  In matrix notation,
  \begin{align*}
    \mZ
    &\coloneqq
      \mW \mX + \vb \vone^{\top}_S
    \\
    &=
      \begin{pmatrix}
        \mW & \vb
      \end{pmatrix}
      \begin{pmatrix}
        \mX \\ \vone^{\top}_S
      \end{pmatrix}
      \coloneqq
      \tilde{\mW}
      \tilde{\mX}\,.
  \end{align*}
  The $\cvec$-Jacobian w.r.t.\,the combined weight is
  \begin{align*}
    \jac^{\cvec}_{\tilde{\mW}}\mZ
    =
    \tilde{\mX}^{\top}
    \otimes
    \mI_{D_{\text{out}}}\,.
  \end{align*}
  In contrast, the $\rvec$-Jacobian is
  \begin{align*}
    \jac^{\cvec}_{\tilde{\mW}}\mZ
    =
    \mI_{D_{\text{out}}}
    \otimes
    \tilde{\mX}^{\top}\,.
  \end{align*}
\end{example}

\switchcolumn[1]
\codeblock{basics/jacobians_shared_linear_layer}

%%% Local Variables:
%%% mode: latex
%%% TeX-master: "../main"
%%% End:


\switchcolumn[0]*
\subsubsection{Hessians, HVPs}
Now that we have covered first-order derivatives, we will move on to second-order derivatives and develop the necessary concepts to understand KFAC, as well as their implementation.
Second-order derivatives are collected into an object called \emph{the Hessian}.
For our purposes, it will be sufficient to consider the Hessian of functions producing a scalar-valued output.
Let's start with the definition of the Hessian of a vector-to-scalar function.

\begin{definition}[Hessian of a vector-to-scalar function]\label{def:vector_hessian}
  Let $\vf: \sR^A \to \sR, \va \mapsto b = f(\va)$ be a vector-to-scalar function.
  Its Hessian $\hess_{\va}b \in \sR^{A \times A}$ collects the second-order partial derivatives into a matrix such that
  \begin{align*}
    [\hess_{\va}b]_{i,j}
    &=
      \frac{\partial^2 b}{\partial [\va]_i \partial [\va]_j}\,.
  \end{align*}
\end{definition}
This definition is limited to functions with vector-valued arguments. The extension to tensor-to-scalar-functions is straightforward; however, it yields a tensor which is less convenient to work with in terms of notation:

\begin{setup}\label{setup:hessians}
  Consider a tensor-to-scalar function $f: \sR^{A_1 \times \dots \times A_N} \to \sR, \tA \mapsto b = f(\tA)$ from a rank-$N$ tensor $\tA$ into a scalar $b$.
\end{setup}

\begin{definition}[General Hessian of a tensor-to-scalar function, \Cref{hessians}]\label{def:general_hessian}
  The general Hessian of $f$ from \Cref{setup:hessians}, $\tH_{\tA}b \in \sR^{A_1 \times \dots \times A_N \times A_1 \times \dots \times A_N}$, collects the second-order partial derivatives of $f$ into a tensor with
  \begin{align*}
    &[\tH_{\tA}b]_{i_1, \dots, i_N, j_1, \dots, j_N}
      \\
    &=
      \frac{\partial^2 b}{\partial [\tA]_{i_1, \dots, i_N} \partial [\tA]_{j_1, \dots, j_N}}\,.
  \end{align*}
\end{definition}
Just like for Jacobians, the Hessian tensor is usually too large to be stored in memory.
Hence, one usually works with it implicitly through matrix-vector products, which can be done without computing the Hessian:

\switchcolumn[1]*
\codeblock{hessian_product}
\switchcolumn[0]

\begin{definition}[Hessian-vector products (HVPs), \Cref{hessian_product}]\label{def:hvp}
  Given a tensor-to-scalar function $f$ from \Cref{setup:hessians} and a tensor $\tV \in \sR^{A_1 \times \dots \times A_N}$ in the input domain, the Hessian-vector product (HVP) $\tU$ of $\tV$ with $\hess$ is the result of the contraction with one of the Hessian's input indices,
  \begin{align*}
    &[\tU]_{i_1, \dots, i_N}
    \\
    &=
      \sum_{j_1, \dots, j_N}
      [\tH_{\tA}b]_{i_1, \dots, i_N, j_1, \dots, j_N} [\tV]_{j_1, \dots, j_N}\,.
  \end{align*}
\end{definition}
For the vector case $N=1$, we have $\tV, \tA, \tH_{\tA}b \to \vv, \va, \hess_{\va}b$ and $\tU \to \vu = \hess_{\va} b$ as suggested by the name.

One way to multiply by the Hessian uses the so-called Pearlmutter trick~\cite{pearlmutter1994fast}.
It relies on the fact that multiplication with higher-order derivatives can be done by nested first-order differentiation.
Hence, multiplication with the Hessian can be done with two VJPs.

\switchcolumn[1]*
\codeblock{hessians}
\switchcolumn[0]

\paragraph{Matricization:} For notational convenience, we will also define matricized versions of the general Hessian from \Cref{def:general_hessian}; the $\cvec$-, and $\rvec$-Hessian. Just like for the Jacobians, it does not matter whether we first flatten the function's input space then compute the Hessian, or compute the general Hessian then matricize it:

\begin{figure}[!h]
  \centering
  \begin{tikzpicture}[%
    font=\scriptsize,%
    thick,
    box/.style = {rectangle, draw=black, rounded corners, fill=VectorGray!50},%%
    ]
    \node[box] (A) at (0,0) {$f: \tA \mapsto b = f(\tA)$};
    \node[box] (B) at (4,0) {$\tJ_{\tA}b$};
    \node[box, align=center] (C) at (0,-2.5) {%
      $f^{\vec} = f \circ \vec^{-1}$\\%
      $\vec(\tA) \coloneq \va \mapsto b$%
    };
    \node[box, align=center] (D) at (4,-2.5) {%
      $\hess_{\va} b$\\%
      $=$\\%
      $\mat(\tH_{\tA}b)$\\%
      $\coloneq$\\%
      $\hess^{\vec}_{\tA}b$%
    };
    \draw[-Stealth] (A.east) -- node[fill=white] {$\tJ$} (B.west);
    \draw[-Stealth] (A.south) -- node[fill=white] {flatten $f$} (C.north);
    \draw[-Stealth] (C.east) -- node[fill=white] {$\jac$} (D.west);
    \draw[-Stealth] (B.south) -- node[fill=white] {matricize $\tH_{\tA}b$} (D.north);
  \end{tikzpicture}
  \caption{Flattening and taking the Hessian commute and lead to the same matricized Hessian.
    $\vec$ denotes one of the flattening conventions from \Cref{def:cvec,def:rvec}.
    $\mat$ denotes matricization and involves two flattenings for the row and column dimensions, respectively.}
\end{figure}


\begin{definition}[$\cvec$-Hessian]\label{def:cvec_hessian}
  For a tensor-to-scalar function $f$ from \Cref{setup:hessians}, the $\cvec$-Hessian $\hess_{\tA}^{\cvec}b \in \sR^{A_N \cdots A_1 \times A_N \cdots A_1}$ results from flattening the input tensor with $\cvec$ and applying the Hessian from \Cref{def:vector_hessian},
  \begin{align*}
    [\hess^{\cvec}_{\tA}b]_{i, j}
    &=
      \frac{\partial^2 b}{\partial [\cvec(\tA)]_{i} \partial [\cvec(\tA)]_{j}}\,.
  \end{align*}
\end{definition}

\begin{definition}[$\rvec$-Hessian]\label{def:rvec_hessian}
  For a tensor-to-scalar function $f$ from \Cref{setup:hessians}, the $\rvec$-Hessian $\hess_{\tA}^{\rvec}b \in \sR^{A_1 \cdots A_N \times A_1 \cdots A_N}$ results from flattening the input tensor with $\rvec$ and applying the Hessian from \Cref{def:vector_hessian},
  \begin{align*}
    [\hess^{\rvec}_{\tA}b]_{i, j}
    &=
      \frac{\partial^2 b}{\partial [\rvec(\tA)]_{i} \partial [\rvec(\tA)]_{j}}\,.
  \end{align*}
\end{definition}

Whenever we consider vector-to-scalar functions, both Hessians are identical and we thus suppress the flattening scheme and write $\hess_{\va}b$. Let's look at some important examples of Hessians that we will come back to later in the text.

\switchcolumn[1]*
\codeblock{hessian_ce_loss}
\switchcolumn[0]

\begin{example}[Softmax cross-entropy loss Hessian, \Cref{hessian_ce_loss}]
  Consider the softmax cross-entropy loss function between the vector-valued logits $\vx \in \sR^C$ and a class label $y \in \{1, \dots, C\}$:
  \begin{align*}
    \ell(\vx, y)
    &=
      -\log([p(\vx)]_y)\,.
  \end{align*}
  with $p(\vx) = \softmax(\vx) \in \sR^C$.
  Its Hessian w.r.t.\,$\vx$ is diagonal-minus-rank-one,
  \begin{align*}
    \hess_{\vx} \ell(\vx, y)
    =
    \diag(p(\vx)) - p(\vx) p(\vx)^\top\,.
  \end{align*}
  See for instance~\cite{dangel2020modular} for a derivation.
\end{example}

\switchcolumn[1]*
\codeblock{hessian_mse_loss}

\switchcolumn[0]
\begin{example}[Square loss Hessian, \Cref{hessian_mse_loss}]
  Consider the square loss function between a vector-valued input $\vx \in \sR^C$ and its associated target $\vy \in \sR^C$:
  \begin{align*}
    \ell(\vx, \vy)
    &=
      \left\lVert
      \vx - \vy
      \right\rVert^2
      \\
      &=
      (\vx - \vy)^{\top} \mI (\vx - \vy)\,.
  \end{align*}
  Its Hessian w.r.t.\,$\vx$ is proportional to the identity,
  \begin{align*}
    \hess_{\vx} \ell(\vx, \vy)
    =
    2 \mI_C\,.
  \end{align*}
\end{example}

We list more important loss function Hessians in \Cref{app:loss_function_hessians}.

%%% Local Variables:
%%% mode: latex
%%% TeX-master: "../main"
%%% End:


\switchcolumn[0]*
\subsubsection{Partial Linearization, Generalized Gauss-Newtons, GGNVPs}\label{sec:partial_linearization}
The last trick for our bag to embark on the KFAC journey is linearization.
It is a useful tool whenever we encounter a composition of functions that we would like to convexify.

Consider for example the function $f = g \circ h$ with $f,g,h: \sR \to \sR$ for simplicity.
We know that convexity is preserved under function composition, so if both $g$ and $h$ are convex, then $f$ will be convex.
But what if only one of the two composites is convex, lets say $g$, but $h$ is not?
Well, then we can replace $h$ with an approximation $h'$ that is convex and approximates the original function somewhat well.
Let's say we are interested in only a neighbourhood around $\vx_0$.
Then, we can obtain a simple, convexified approximation $f' \approx f$ in that neighbourhood by linearizing $h$ around $\vx_0$, resulting in a function $(\lin_{\vx_0}(h))(\vx)$.
This involves a first-order Taylor approximation:

\switchcolumn[1]*
\codeblock{basics/linearization}
\switchcolumn[0]

\begin{definition}[Linearization (vector case, \Cref{linearization})]\label{def:vector_linearization}
  Consider a vector-to-vector function $f$ from \Cref{setup:vector_to_vector_function}.
  The linearization of $f$ at an anchor point $\va_0 \in \sR^A$ denoted by $\lin_{\va_0}(f): \sR^A \to \sR^B$ is its first-order Taylor expansion,
  \begin{align*}
    (\lin_{\va_0}(f))(\va) = f(\va_0) + \jac_{\va_0}f(\va_0) (\va - \va_0)\,,
  \end{align*}
  with the Jacobian from \Cref{def:vector_jacobian}.
  Note that $\lin_{\va_0}(f)$ is linear in $\va$ and coincides with the original function at the anchor point, $(\lin_{\va_0}(f))(\va_0) = f(\va_0)$ (the Jacobians also concide).
\end{definition}

\begin{definition}[Linearization (tensor case, \Cref{linearization})]\label{def:tensor_linearization}
  The linearization of a tensor-to-tensor function from \Cref{setup:jacobians} at an anchor point $\tA_0 \in \sR^{A_1 \times \ldots \times A_N}$, denoted by $\lin_{\tA_0}(f)$ is defined per-entry as
  \begin{align*}
     & \left[
      (\lin_{\tA_0}(f))(\tA)
      \right]_{i_1, \ldots, i_M} = f(\tA_0)
    \\
     & \hspace{2em}+
    \sum_{j_1, \ldots, j_N}
    \left[
      \jac_{\tA_0}f(\tA_0)
      \right]_{i_1, \ldots, i_M, j_1, \ldots, j_N}
    \\
     & \hspace{6.2em}\left[
      \tA - \tA_0
      \right]_{j_1, \ldots, j_N}\,,
  \end{align*}
  with the Jacobian from \Cref{def:general_jacobian}. Note that this is nothing else but the function evaluated at the anchor point plus the JVP (\Cref{def:jvp}) with the distance to the anchor.
\end{definition}

In deep learning, we will often face the situation where $f = g \circ h, \tA \mapsto b = f(\tA)$ and $h$ is non-convex while $g$ is convex.
This means the Hessian of $f$ can be indefinite; but for algorithms we require a positive definite approximation to the Hessian.
We can obtain that by considering the partially linearized function $\bar{f} = g \circ \lin_{\tA_0}(h)$, whose Hessian is positive semi-definite. The Hessian of this partially linearized function is called the generalized Gauss-newton (GGN) matrix.

Let's stick to our one-dimensional example for a moment, i.e.\,let $f(x) = (g \circ h)(x) \in \sR$. If we use the chain rule twice, we obtain the following expression for the Hessian:
\begin{align*}
  \hess_x (g \circ h)(x) =
  \jac_x h(x) \cdot \hess_{h(x)} g(h(x)) \cdot \jac_x h(x)
  \\
  +
  \hess_x h(x) \cdot \jac_{h(x)} g(h(x))\,.
\end{align*}
Now, if we take the hessian of the partially linearized function $\bar{f}(x) = (g \circ \lin_{x_0}(h))(x)$, and using the shorthand $\bar{h} = \lin_{x_0}(h)$ the second term disappears as the linear function's Hessian is zero:
\begin{align*}
  \hess_x (g \circ \bar{h})(x) =
  \jac_x \bar{h}(x) \cdot \hess_{\bar{h}(x)} g(\bar{h}(x)) \cdot \jac_x \bar{h}(x)
  \\
  +
  \underbrace{\hess_x \bar{h}(x)}_{= 0} \cdot \jac_{\bar{h}(x)} g(\bar{h}(x))\,.
\end{align*}
If we evaluate both equations at the anchor point, setting $x = x_0$, we obtain that the first terms coincide.

\begin{figure}[!h]
  \centering
  \begin{tikzpicture}[%
      font=\scriptsize,%
      thick,
      box/.style = {rectangle, draw=black, rounded corners, fill=VectorGray!50},%%
    ]
    \node[box] (A) at (0,0) {$f = g \circ h: \tA \mapsto \tB \mapsto c$};
    \node[box] (B) at (4,0) {$\tH_{\tA}c$};
    \node[box, align=center] (C) at (0,-2.5) {%
      $\bar{f}_{\tA_0} = g \circ \lin_{\tA_0} h$\\%
    };
    \node[box, align=center] (D) at (4,-2.5) {%
    $\tG_{\tA}f(\tA)|_{\tA_0 = \tA}$
    \\
    $=$\\%
    $\tH_{\tA} \bar{f}_{\tA_0}(\tA)|_{\tA_0 = \tA}$\\%
    };
    \draw[-Stealth] (A.east) -- node[fill=white] {$\tH$} (B.west);
    \draw[-Stealth] (A.south) -- node[fill=white, align=center] {partially linearize $h$ \\
      around some $\tA_0$} (C.north);
    \draw[-Stealth] (C.east) -- node[fill=white, align=center] {$\tH$, then\\ set\\ $\tA_0 = \tA$} (D.west);
    \draw[-Stealth] (B.south) -- node[fill=white, align=center] {ignore second-order\\ effects of $h$} (D.north);
  \end{tikzpicture}
  \caption{Taking the Hessian and partial linearization commute and lead to the same GGN. All operations also commute with flattening the function versus matrizicing the Hessian tensor, so we can apply any flattening scheme on top to obtain the $\rvec$- and $\cvec$-GGN matrices.}
\end{figure}

We will not use the linearization aspect explicitly in the computation, but instead rely on GGN-vector products which can be built from the previously introduced autodiff operations.
We first provide the definition of the GGN in the matrix and tensor case, before we show how to compute the GGN-vector product.

\switchcolumn[1]*
\codeblock{basics/ggns}
\switchcolumn[0]

\begin{setup}[Composite vector-to-vector-to-scalar function]\label{setup:composite_vector_to_vector_to_scalar_function}
  Let
  \begin{align*}
    f: \sR^{A} & \to \sR
    \\
    \va        & \mapsto c = f(\va)
  \end{align*}
  be the composite of a vector-to-vector function $h$ and a vector-to-scalar function $g$, that is
  \begin{align*}
    f = g \circ h: \sR^A & \to \sR^B \to \sR
    \\
    \va                  & \mapsto \vb = h(\va) \mapsto c = g(\vb)\,.
  \end{align*}
\end{setup}

\begin{definition}[Generalized Gauss-Newton (GGN) matrix (vector case)]\label{def:vector_ggn}
  The GGN matrix of a vector-to-vector-to-scalar function $f$ from \Cref{setup:composite_vector_to_vector_to_scalar_function}, $\ggn_{\va} f(\va) \in \sR^{A \times A}$ is
  \begin{align*}
    \ggn_{\va} f(\va)
    =
    (\jac_{\va} \vb)^{\top}
    (\hess_{\vb} c)
    (\jac_{\va} \vb)\,,
  \end{align*}
  i.e.\,the second composite's Hessian, left- and right-multiplied with the first composite's Jacobian.
\end{definition}

\switchcolumn[1]*
\codeblock{basics/ggn_rosenbrock}
\switchcolumn[0]

\begin{example}[GGN for the Rosenbrock function, \Cref{ggn_rosenbrock}]
  Consider the 2d Rosenbrock function $f_{\alpha}: \sR^2 \to \sR$ with
  \begin{align*}
    f_{\alpha}(\vx)
    =
    (1 - x_1)^2 + \alpha (x_2 - x_1^2)^2\,.
  \end{align*}
  with some $\alpha > 0$.
  We can express $f_{\alpha} = c \circ g_{\alpha}$,\footnote{\url{https://www.brnt.eu/phd/node10.html}}
  \begin{align*}
    g_{\alpha}(\vx) = \begin{pmatrix}
                        1 - x_1 \\
                        \sqrt{\alpha} (x_2 - x_1^2)
                      \end{pmatrix}
    \shortintertext{and convex}
    c(\vg) = \vg^\top \vg\,,
  \end{align*}
  namely square loss.

  Linearizing $g_{\alpha}$ w.r.t. $\vx$ around $\vx'$ gives
  \begin{align*}
    g^{\text{lin}}_{\alpha}(\vx) = g_{\alpha}(\vx') + (\jac_{\vx'}g_{\alpha}(\vx')) (\vx - \vx')\,.
  \end{align*}
  with
  \begin{align*}
    \jac_{\vx}g_{\alpha}(\vx)
    =
    \begin{pmatrix}
      -1                   & 0             \\
      -2 \sqrt{\alpha} x_1 & \sqrt{\alpha}
    \end{pmatrix}\,.
  \end{align*}
  To form the GGN of $f_{\alpha} = c \circ g_{\alpha}$ at $\vx = \vx'$, we take the Hessian of $f_{\alpha}' = c \circ g_{\alpha}^{\text{lin}}$, and evaluate it at $\vx = \vx'$,
  \begin{align*}
      & (\hess_{\vx} f_{\alpha}'(\vx))|_{x=x'}
    \\
    = & 2 (\jac_{\vx} g_{\alpha}(\vx))|_{x=x'}^{\top}
    (\jac_{\vx} g_{\alpha}(\vx))|_{x=x'}
    \\
    = & 2
    \begin{pmatrix}
      1 + 4\alpha x_1^2 & -2 \alpha x_1 \\
      -2 \alpha x_1     & \alpha
    \end{pmatrix}\,.
  \end{align*}
  This matrix is PSD because it is an outer product of two matrices.
  It is also different from the Hessian matrix
  \begin{align*}
    \hess_{\vx} f_{\alpha}(\vx)
    =
    \begin{pmatrix}
      2 + 12 \alpha x_1^2 - 4 \alpha x_2 & -4 \alpha x_1 \\
      -4 \alpha x_1                      & 2 \alpha
    \end{pmatrix}\,.
  \end{align*}
  \begin{figure}[H]
    \centering
    \includegraphics[width=\linewidth]{../kfs/plots/linearized_rosenbrock.pdf}
    \caption{The 2d Rosenbrock function $f_{\alpha=10}$ (solid contour lines) with and its approximation $f'_{\alpha=10, \vx'} = \ell \circ g^{\text{lin}}_{\alpha=10, \vx'}$ (dashed contour lines) when partially linearizing it around an anchor $\vx'$ (star).}
  \end{figure}
\end{example}

\begin{setup}[Composite tensor-to-tensor-to-scalar function]\label{setup:composite_tensor_to_tensor_to_scalar_function}
  Let the tensor-to-scalar function
  \begin{align*}
    f: \sR^{A_1 \times \ldots \times A_N} & \to \sR
    \\
    \tA                                   & \mapsto c = f(\tA)
  \end{align*}
  be the composite of a tensor-to-tensor function $h$ and a tensor-to-scalar function $g$, that is
  \begin{align*}
    f = g \circ h: & \sR^{A_1 \times \ldots \times A_N} \to \sR^{B_1 \times \ldots \times B_M}  \to \sR
    \\
                   & \tA \mapsto \tB = h(\tA) \mapsto c = g(\tB)\,.
  \end{align*}
\end{setup}

\begin{definition}[Generalized Gauss-Newton (GGN) tensor (\Cref{ggns})]\label{def:general_ggn}%
  The GGN tensor of a tensor-to-tensor-to-scalar function $f$ from \Cref{setup:composite_tensor_to_tensor_to_scalar_function}, $\tG_{\tA} f(\tA) \in \sR^{A_1 \times \ldots \times A_N \times A_1 \times \ldots \times A_N}$ is the Hessian of the partially linearized function $\bar{f} = g \circ \lin_{\tA_0}(h)$, evaluated at the anchor point $\tA_0 = \tA$.
  \begin{align*}
    \tG_{\tA} f(\tA)
    =
    \tH_{\tA} \bar{f}(\tA)|_{\tA_0 = \tA}\,.
  \end{align*}
  where $\bar{f} = g \circ \lin_{\tA_0}(h)$.

  We can express this tensor as a tensor contraction between the Jacobians and Hessians of the two composites, i.e.\,
  \begin{align*}
    [\tG_{\tA} f(\tA)]_{i_1, \ldots, i_N, j_1, \ldots, j_N}
    \\
    =
    \textcolor{VectorPink}{\sum_{k_1, \dots, k_M}}
    \textcolor{VectorBlue}{\sum_{l_1, \dots, l_M}}
     & [\jac_{\tA} \tB]_{\textcolor{VectorPink}{k_1, \dots, k_M}, i_1, \dots, i_N}
    \\
     & [\hess_{\tB} c]_{\textcolor{VectorPink}{k_{1}, \ldots, k_{M}}, \textcolor{VectorBlue}{l_{1}, \ldots, l_{M}}}
    \\
     & [\jac_{\tA} \tB]_{\textcolor{VectorBlue}{l_1, \ldots, l_M}, j_1, \ldots, j_N}\,.
  \end{align*}
\end{definition}
This expression seems daunting at first, so we will convert things back to matrix notation very soon. Before doing that, let's introduce multiplication with the GGN:

\switchcolumn[1]*
\codeblock{basics/ggn_product}
\switchcolumn[0]

\begin{definition}[GGN-vector-product (GGNVP, \Cref{ggn_product})]\label{def:ggnvp}%
  Consider the GGN tensor of a tensor-to-tensor-to-scalar function $f$ from \Cref{setup:composite_tensor_to_tensor_to_scalar_function,def:general_ggn}.
  The GGN-vector-product (GGNVP) $\tU \in \sR^{A_1 \times \ldots \times A_N}$ with a tensor $\tV \in \sR^{A_1 \times \ldots \times A_N}$ from the input domain is
  \begin{align*}
     & [\tU]_{i_1, \dots, i_N}
    \\
     & =
    \sum_{j_1, \dots, j_N}
    [\tG_{\tA} f(\tA)]_{i_1, \dots, i_N, j_1, \dots, j_N}
    [\tV]_{j_1, \dots, j_N}\,,
  \end{align*}
  and decomposes into a JVP, HVP, and JVP when applying the composition from \Cref{def:general_ggn}.
\end{definition}
This is easiest to see for the vector-to-vector-to-scalar case where $\tA, \tB, \tU, \tV \to \va, \vb, \vu, \vv$ and $\vu = (\jac_{\va}\vb)^{\top} (\hess_{\vb} c) (\jac_{\va} \vb) \vv$, which can be written as a matrix chain and computed without explicitly building up any of the matrices in memory~\cite{schraudolph2002fast}.

\paragraph{Matricization} As for the Hessian and Jacobian, we can flatten both composite functions before applying the partial linearization and taking the Hessian to obtain the GGN.
This is equivalent to matricizing the GGN tensor from \Cref{def:general_ggn}.
And it is also equivalent to matricizing the Hessians and Jacobians in the general definition.

\begin{definition}[$\cvec$ and $\rvec$ GGN matrices, \Cref{ggns}]\label{def:vec_ggns}
  For a tensor-to-tensor-to-scalar function $f$ from \Cref{setup:composite_tensor_to_tensor_to_scalar_function}, we define the $\cvec$ and $\rvec$ GGN matrices by flattening the composite functions before applying the partial linearization and taking the Hessian. This yields the flattened GGN matrices $\ggn^{\vec}_{\tA}f(\tA) \in \sR^{A_1 \cdots A_N \times A_1 \cdots A_N}$ where $\vec \in \{\cvec, \rvec\}$ which can be written as matrix chain
  \begin{align*}
    \ggn^{\vec}_{\tA} f(\tA)
    =
    (\jac^{\vec}_{\tA} \tB)^{\top}
    (\hess^{\vec}_{\tB} c)
    (\jac^{\vec}_{\tA} \tB)\,.
  \end{align*}
  using the previously defined flattened Hessians (\Cref{def:cvec_hessian,def:cvec_hessian}) and Jacobians (\Cref{def:cvec_jacobian,def:rvec_jacobian}).
\end{definition}

Again, it is important to emphasize that the matrix GGN depends on the flattening scheme.
To emphasize this point, we conclude this section with the following:

\switchcolumn[1]*
\codeblock{basics/ggns_linear_regression}
\switchcolumn[0]

\begin{example}[GGN of linear regression (\Cref{ggns_linear_regression})]
  Consider the least squares objective
  \begin{align*}
    \gL(\mW) = \sum_n \ell_n(\mW)
  \end{align*}
  where $\ell_n = c_n \circ f_n$ is the composition of a linear classifier and square loss
  on a data point labeled $n$, that is
  $f_n(\mW) = \mW \vx_n$ and $c_n(\vz) = \frac{1}{2}\left\lVert \vz - \vy_n \right\rVert^2_2$.

  Using the shorthands $\vf_n \coloneq f_n(\mW)$ and $c_n \coloneq c_n(\vf_n)$, the matrix GGNs are given by
  \begin{align*}
    \ggn_{\mW}^{\vec}\gL(\mW)
    \coloneq
    \sum_n
    \ggn_{\mW}^{\vec} \ell_n(\mW)
    \\
    =
    \sum_n
    (\jac^{\vec}_{\mW} \vf_n)^{\top}
    (\hess^{\vec}_{\vf_n} c_n)
    (\jac^{\vec}_{\mW} \vf_n)\,.
  \end{align*}
  We can use the results from previous examples, specifically \Cref{jacobians_linear_layer,ex:square_loss_hessian}, to obtain the following expressions
  \begin{align*}
    \ggn_{\mW}^{\cvec}\gL(\mW)
     & =
    \left(
    \sum_n \vx_n \vx_n^{\top}
    \right)
    \otimes \mI
    \\
    \ggn_{\mW}^{\rvec}\gL(\mW)
     & =
    \mI
    \otimes
    \left(
    \sum_n \vx_n \vx_n^{\top}
    \right)
  \end{align*}
\end{example}
Two interesting observations about this result are (i) both GGNs are a Kronecker product (ii) whose order of factors depends on the flattening scheme we are using.
This already hints at using a Kronecker product to approximate the exact GGN, which is what KFAC aims to do.


%%% Local Variables:
%%% mode: latex
%%% TeX-master: "../main"
%%% End:


\switchcolumn[0]*
\subsection{Curvature Matrices in Deep Learning}\label{subsec:curvature-matrices}

The previous section introduced Jacobians, Hessians, partial linearizations and the resulting generalized Gauss-Newton (GGN) objects in the language of automatic differentiation for arbitrary functions.
Here, we will switch gears and quickly walk through how these matrices look like in the context of deep learning.

\switchcolumn[1]
\codeblock{basics/hessian_factorizations}
\switchcolumn[0]

\paragraph{Parameter list/tuple format:} One new aspect we have to deal with is that ML libraries like PyTorch represent parameters as lists/tuples of variables.
We consider a neural network $f(\vtheta, x): \Theta \times \gX \to \gF$ with parameters in list/tuple-format,
\begin{align*}
  \vtheta = (\vtheta^{(1)}, \vtheta^{(2)}, \ldots, \vtheta^{(L)}),
\end{align*}
where each $\vtheta^{(i)}$ is an arbitrary tensor. To be able to use matrix expressions, we will often consider the concatenation of flattened vectors,
\begin{align*}
  \vec(\vtheta)
  =
  \begin{pmatrix}
    \vec(\vtheta^{(1)}) \\
    \vec(\vtheta^{(2)}) \\
    \vdots              \\
    \vec(\vtheta^{(L)})
  \end{pmatrix}
  \in \sR^D
  \,,
\end{align*}
where $\vec \in \{ \rvec, \cvec \}$ is one of the previously described flattening operations.
This convention allows $\vec$ to handle parameters in list/tuple format, and is a simple generalization which applies the original definition to each element of the list/tuple and concatenates all flattened items.
However, in code we will still work with the list/tuple format.

\paragraph{Empirical risk:} We consider a data set $\sD = \{(\vx_n, \vy_n) \in \gX \times \gY \mid n = 1, \dots, N \}$ containing $N$ independent and identically distributed (i.i.d.)\,samples.
The inputs are processed by the neural network $f: \Theta \times \gX \to \gF$ and the resulting predictions are scored with a criterion function $c: \gF \times \gY \to \sR$, such as mean-squared or softmax cross-entropy loss.
For a datum $n$, we define the per-datum loss as:
\begin{align*}
  (\ell_n: \Theta \to \sR) = (c_n \circ f_n: \Theta \to \gF \to \sR)
  \\
  \ell_n(\vtheta) = c_n(f_n(\vtheta))
\end{align*}
where $c_n(\bullet) \coloneq c(\bullet, \vy_n)$ and $f_n(\bullet) \coloneq f(\bullet, \vx_n)$.
We will often use the shorthands $c_n, f_n, \ell_n$ for the per-datum criteria, predictions, and losses, respectively.

\paragraph{Reduction factor:} The per-datum losses $\{\ell_1, \dots, \ell_N\}$ are accumulated into a single scalar which yields the empirical risk
\begin{align*}
  \gL_{\sD}: \Theta & \mapsto \sR
  \\
  \vtheta           & \mapsto \gL_{\sD}(\vtheta) = R \sum_{n=1}^N \ell_n(\vtheta) = R \sum_{n=1}^N c_n(f_n(\vtheta))
\end{align*}
where $R$ is the reduction factor. Common values for $R$ are $\nicefrac{1}{N}, 1$ and $\nicefrac{1}{N \dim(\gF)}$.
For the purpose of this text, $\sD$ can be any collection of data points, e.g.\,the full data set or a mini-batch.

\subsubsection{The Hessian}\label{sec:basics_dl_hessian}
Consider the empirical risk from above. The Hessian matrix w.r.t.\,to a flattening scheme $\vec$ is given by
\begin{align*}
  \hess_{\vtheta}^{\vec} \gL_{\sD}(\vtheta)
  =
  R
  \sum_{n=1}^N
  \hess_{\vtheta}^{\vec} \ell_n(\vtheta) \in \sR^{D \times D}\,.
\end{align*}
and contains the second-order partial derivatives of $\gL_{\sD}(\vtheta)$, that is
\begin{align*}
  [\hess_{\vtheta}^{\vec} \gL_{\sD}(\vtheta)]_{i,j}
  =
  \frac{\partial^2 \gL_{\sD}(\vtheta)}{\partial [\vec(\vtheta)]_i \partial [\vec(\vtheta)]_j}\,.
\end{align*}

\switchcolumn[1]*
\includegraphics{../kfs/plots/synthetic_cvec_hessian.pdf}

\includegraphics{../kfs/plots/synthetic_rvec_hessian.pdf}
\switchcolumn[0]

Due to the tuple/list structure of $\vtheta$, the Hessian has a block structure,
\begin{align*}
  \hess_{\vtheta}^{\vec} \gL_{\sD}(\vtheta)
  =
  \begin{pmatrix}
    \hess_{\vtheta^{(1)}}^{\vec} \gL_{\sD}(\vtheta)
    &
      \hess_{\vtheta^{(1)}, \vtheta^{(2)}}^{\vec} \gL_{\sD}(\vtheta)
    &
      \cdots
    &
      \hess_{\vtheta^{(1)}, \vtheta^{(L)}}^{\vec} \gL_{\sD}(\vtheta)
    \\
    \hess_{\vtheta^{(2)}, \vtheta^{(1)}}^{\vec} \gL_{\sD}(\vtheta)
    &
      \hess_{\vtheta^{(2)}}^{\vec} \gL_{\sD}(\vtheta)
    &
      \cdots
    &
      \hess_{\vtheta^{(2)}, \vtheta^{(L)}}^{\vec} \gL_{\sD}(\vtheta)
    \\
    \vdots & \cdots & \ddots & \vdots
    \\
    \hess_{\vtheta^{(L)}, \vtheta^{(1)}}^{\vec} \gL_{\sD}(\vtheta)
    &
      \hess_{\vtheta^{(L)}, \vtheta^{(2)}}^{\vec} \gL_{\sD}(\vtheta)
    &
      \cdots
    &
      \hess_{\vtheta^{(L)}}^{\vec} \gL_{\sD}(\vtheta)
  \end{pmatrix}\,.
\end{align*}
where $\hess_{\vtheta^{(i)}, \vtheta^{(j)}}^{\vec} \gL_{\sD}(\vtheta)$ contains mixed second-order partial derivatives w.r.t.\,$\vtheta^{(i)}$ and $\vtheta^{(j)}$.

In the following, we will only consider the block diagonal approximation of this matrix,
\begin{align*}
  \tilde{\hess}_{\vtheta}^{\vec} \gL_{\sD}(\vtheta)
  =
  \begin{pmatrix}
    \hess_{\vtheta^{(1)}}^{\vec} \gL_{\sD}(\vtheta)
    &
      \vzero
    &
      \cdots
    &
      \cdots
    &
      \vzero
    \\
    \vzero
    &
      \hess_{\vtheta^{(2)}}^{\vec} \gL_{\sD}(\vtheta)
    &
      \vzero
    &
      \cdots
    &
      \vzero
    \\
    \vdots & \cdots & \ddots & \vdots
    \\
    \vzero
    &
      \vzero
    &
      \vzero
    &
      \cdots
      &
      \hess_{\vtheta^{(L)}}^{\vec} \gL_{\sD}(\vtheta)
  \end{pmatrix}\,.
\end{align*}
i.e., individual blocks $\{ \hess_{\vtheta^{(i)}}^{\vec} \gL_{\sD}(\vtheta)$.

%%% Local Variables:
%%% mode: latex
%%% TeX-master: "../main"
%%% End:


\subsubsection{The Generalized Gauss-Newton (GGN)}
As previously described, the generalized Gauss-Newton matrix arises from the partial linearization of a composite function.
In our case, we will linearize the neural network predictions $f_n \to f_n^{\text{lin}} = \lin_{\vtheta_0}(f_n)$, then form the partially-linearized per-sample losses $\tilde{\ell}_n = c_n \circ f_n^{\text{lin}}$ which themselves give rise to the partially linearized empirical risk $\tilde{\gL}_{\sD}(\vtheta) = R \sum_{n=1}^{N} \tilde{\ell}_n(\vtheta)$.
We then define the vec GGN matrix of $\gL_{\sD}(\vtheta)$ as the Hessian of the partially-linearized empirical risk where the linearization is anchored at the evaluation point of the GGN, just like in \Cref{sec:partial_linearization},
\begin{align*}
  \ggn^{\vec}_{\vtheta} \gL_{\sD}(\vtheta)
  &=
  \left.\hess_{\vtheta}^{\vec} \tilde{\gL}_{\sD}(\vtheta)\right|_{\vtheta_0 = \vtheta} \in \sR^{D \times D}
    \\
  &=
    R \sum_{n=1}^N
    \left.\hess_{\vtheta}^{\vec} \tilde{\ell}_n(\vtheta)\right|_{\vtheta_0 = \vtheta},
\end{align*}
where $\vec$ is one of the flattening operations.
We can also express the GGN through Jacobians,
\begin{align*}
  &\ggn^{\vec}_{\vtheta} \gL_{\sD}(\vtheta)
  \\
  &=
  R \sum_{n=1}^N
  [\jac_{\vtheta}^{\vec} f_n]^{\top}
  [\hess^{\vec}_{f_n} \underbrace{c_n(f_n)}_{\ell_n}]
  [\jac_{\vtheta}^{\vec} f_n].
\end{align*}

\switchcolumn[1]
\begin{figure}
  \centering
  \begin{minipage}[t]{0.495\linewidth}
    \centering
    $\cvec$\vspace{1ex}
    \includegraphics[width=\linewidth]{../kfs/plots/synthetic_cvec_ggn.pdf}
  \end{minipage}
  \hfill
  \begin{minipage}[t]{0.495\linewidth}
    \centering
    $\rvec$\vspace{1ex}
    \includegraphics[width=\linewidth]{../kfs/plots/synthetic_rvec_ggn.pdf}
  \end{minipage}
  \\
  \begin{minipage}[t]{0.495\linewidth}
    \centering
    \includegraphics[width=\linewidth]{../kfs/plots/synthetic_cvec_ggn_bda.pdf}
  \end{minipage}
  \hfill
  \begin{minipage}[t]{0.495\linewidth}
    \centering
    \includegraphics[width=\linewidth]{../kfs/plots/synthetic_rvec_ggn_bda.pdf}
  \end{minipage}
  \caption{Visualization of the GGN and its block-diagonal approximation using different flattening schemes.
    The GGN blocks are visually highlighted with white lines.
    Left column uses $\cvec$-flattening, right column uses $\rvec$-flattening.
    The GGN was evaluated on synthetic data ($N=100$) using an MLP with three fully-connected layers and ReLU activations (5-4-4-3, our notation considers applying the weight matrix and adding the bias as two layers, hence $L=6$) and square loss.
  }
\end{figure}

\switchcolumn[0]
Just like the Hessian, the GGN has a block structure. With the abbreviations
$\ggn_{k,l}^{\vec}\gL \coloneq \ggn^{\vec}_{\vtheta^{(k)}, \vtheta^{(l)}} \gL_{\sD}(\vtheta) = \hess^{\vec}_{\vtheta^{(k)}, \vtheta^{(l)}} \tilde{\gL}_{\sD}(\vtheta) = R \sum_{n=1}^N [\jac_{\vtheta^{(k)}}^{\vec} f_n]^{\top} [\hess^{\vec}_{f_n} c_n(f_n)] [\jac_{\vtheta^{(l)}}^{\vec} f_n]$ and $\ggn_{k}^{\vec}\gL \coloneq \ggn^{\vec}_{\vtheta^{(k)}} \gL_{\sD}(\vtheta)$, we can write the GGN as
\begin{align*}
  \ggn_{\vtheta}^{\vec} \gL
  =
  \begin{pmatrix}
    \ggn_1^{\vec} \gL
    &
      \ggn_{1, 2}^{\vec} \gL
    &
      \cdots
    &
      \ggn_{1, L}^{\vec} \gL
    \\
    \ggn_{2, 1}^{\vec} \gL
    &
      \ggn_2^{\vec} \gL
    &
      \cdots
    &
      \ggn_{2, L}^{\vec} \gL
    \\
    \vdots & \cdots & \ddots & \vdots
    \\
    \ggn_{L, 1}^{\vec} \gL
    &
      \ggn_{L, 2}^{\vec} \gL
    &
      \cdots
    &
      \ggn_L^{\vec} \gL
  \end{pmatrix}\,.
\end{align*}
In the following, we will only consider the block diagonal approximation of this matrix,
\begin{align*}
  \tilde{\ggn}_{\vtheta}^{\vec} \gL
  =
  \begin{pmatrix}
    \ggn_1^{\vec} \gL & \vzero & \cdots & \vzero
    \\
    \vzero & \ggn_2^{\vec} \gL & \ddots & \vdots
    \\
    \vdots & \ddots & \ddots & \vzero
    \\
    \vzero & \cdots & \vzero & \ggn_L^{\vec} \gL
  \end{pmatrix}\,,
\end{align*}
i.e., individual blocks $\{ \ggn_{\vtheta^{(k)}}^{\vec} \gL_{\sD}(\vtheta)\}_{k=1}^L$.

\paragraph{The GGN as a self-outer product}
Let us look at the GGN contributed by a single datum and suppress the index $_n$ for now, as well as the reduction factor $R$.
This contribution is
\begin{align*}
  \underbrace{[\jac_{\vtheta}^{\vec} f]^{\top}}_{\dim(\Theta) \times \dim(\gF)}
  \underbrace{[\hess^{\vec}_{f} c(f)]}_{\dim(\gF) \times \dim(\gF)}
  \underbrace{[\jac_{\vtheta}^{\vec} f]}_{\dim(\gF) \times \dim(\Theta)}\,.
\end{align*}
We will now make this more symmetric.
Note that, by assumption, the criterion function $c$ is convex in $f$.
This means that the flattened Hessian $\hess^{\vec}_{f} c(f)$ is positive semi-definite.
Since any positive semi-definite matrix $\mA \in \sR^{C \times C}$ can be expressed as an outer product $\mA = \mB \mB^{\top}$ where $\mB \in \sR^{C \times \rank(\mA)}$, we can find a symmetric factorization $\mS^{\vec} \in \sR^{\dim(\gF) \times \dim(\gF)}$ of the criterion's Hessian such that $\hess^{\vec}_{f} c(f) = \mS^{\vec} {\mS^{\vec}}^{\top}$.
With that, we can then write the upper expression as
\begin{align*}
  &[\jac_{\vtheta}^{\vec} f]^{\top} \mS^{\vec} {\mS^{\vec}}^{\top} [\jac_{\vtheta}^{\vec} f]
  \\
  &=
  \underbrace{([\jac_{\vtheta}^{\vec} f]^{\top} \mS^{\vec})}_{\coloneq \mV^{\vec} \in \sR^{\dim(\Theta) \times \dim(\gF)}}
  ([\jac_{\vtheta}^{\vec} f]^{\top} \mS^{\vec})^{\top}
  \\
  &=
  \mV^{\vec} {\mV^{\vec}}^{\top}\,.
\end{align*}
I.e., we can express the flattened GGN contribution of a single datum as a self-outer product.
In the following, we will not look further into $\mV^{\vec}$; see \cite{dangel2022vivit,papyan2019measurements} for a detailed discussion. Instead, we want to derive the factorization for the Hessian of the square and cross-entropy losses, which will become useful in later chapters.
\begin{example}[Symmetric factorization of the Hessian of square loss]\label{ex:mseloss_hessian_factorization}
  Consider the square loss $c$ from \Cref{ex:square_loss} and its Hessian from \Cref{ex:square_loss_hessian}.
  The Hessian's symmetric factorization is simply
  \begin{align*}
    \mS^{\vec} = \mI
  \end{align*}
  because $\mS^{\vec} {\mS^{\vec}}^{\top} = \mI = \hess_{f}^{\vec}c$.
\end{example}

\begin{example}[Symmetric Hessian decomposition of softmax cross-entropy loss, \Cref{hessian_factorizations}]\label{ex:crossentropyloss_hessian_factorization}
  The Hessian's symmetric factorization is \citep[e.g.][]{papyan2019measurements}
  \begin{align*}
    \mS^{\vec} = \diag(\sqrt{\vp}) - \vp \sqrt{\vp}^{\top}
  \end{align*}
  where $\vp = \softmax(\vf)$ and the square root is applied element-wise.
  To see this, we can form $\mS^{\vec} {\mS^{\vec}}^{\top}$ which yields
  \begin{align*}
    \diag(\vp) - 2 \vp \vp^{\top} + \vp \sqrt{\vp}^{\top} \sqrt{\vp} \vp^{\top}
    \\
    = \diag(\vp) - \vp \vp^{\top}
  \end{align*}
  using the fact that $\vp^{\top} \vone = 1 = \sqrt{\vp}^{\top} \sqrt{\vp}$.
  This expression equals the Hessian from \Cref{ex:square_loss_hessian}.
\end{example}
Note that the Hessian factorization for both the square loss and the softmax cross-entropy loss do not depend on the label, i.e., the second argument of the criterion function.
We will make use of this property in the next section to connect the GGN with the Fisher matrix for regression and classification tasks.

\switchcolumn[1]
\codeblock{basics/hessian_factorizations}
\switchcolumn[0]
%%% Local Variables:
%%% mode: latex
%%% TeX-master: "../main"
%%% End:


\subsubsection{The Fisher}\label{sec:fisher}
Probabilistic perspective

Explain type-1 versus type-2
%%% Local Variables:
%%% mode: latex
%%% TeX-master: "../main"
%%% End:


\subsubsection{The Connection between GGN \& Fisher}
The GGN, type-I and type-II Fisher can all be written as weighted sum over matrices sandwiched between the per-sample Jacobians (remember that we can mentally set $-\log r(\rvy \mid \vf_n) = c(\vf_n, \rvy)$ for square and softmax cross-entropy loss):
\begin{align*}
  \mG(\vtheta)
   & =
  R \sum_n
  \begin{aligned}[t]
     & (\jac_{\vtheta}\vf_n)^\top                           \\
     & \textcolor{VectorBlue}{\hess_{\vf_n}c(\vf_n, \vy_n)} \\
     & \jac_{\vtheta}\vf_n
  \end{aligned}
  \\
  \mF^{\text{II}}(\vtheta)
   & =
  R \sum_n
  \begin{aligned}[t]
     & (\jac_{\vtheta}\vf_n)^\top                                                                 \\
     & \textcolor{VectorPink}{\E_{r(\rvy \mid \vf_n)}[-\hess_{\vf_n} \log( r(\rvy \mid \vf_n)) ]} \\
     & \jac_{\vtheta}\vf_n
  \end{aligned}
  \\
  \mF^{\text{I}}(\vtheta)
   & =
  R \sum_n
  \begin{aligned}[t]
     & (\jac_{\vtheta}\vf_n)^\top                          \\
     & \textcolor{VectorTeal}{\E_{r(\rvy \mid \vf_n)}[
    \begin{aligned}[t]
       & -\nabla_{\vf_n} \log( r(\rvy \mid \vf_n))           \\
       & (-\nabla_{\vf_n} \log( r(\rvy \mid \vf_n)))^{\top}]
    \end{aligned}} \\
     & \jac_{\vtheta}\vf_n
  \end{aligned}
\end{align*}
In previous sections, we already showed that for square loss and softmax cross-entropy the criterion function's Hessian $\hess_\vf c(\vf, \vy) = -\hess_{\vf} \log( r(\rvy = \vy \mid \vf)$ does not depend on the value of the target random variable $\rvy$!
Therefore, the expectation in the type-II Fisher effectively disappears and we are free to set $\rvy = \vy_n$ because this does not change the Hessian.
This means the type-II Fisher is equivalent to the GGN for square loss and softmax cross-entropy.
Note that we cannot drop the expectation in the type-I Fisher, though.\footnote{This is precisely why we needed a separate definition for the Monte-Carlo-approximated type-I Fisher,
as it yields different values from the type-II Fisher even for the square and softmax cross-entropy losses.}
But from the equivalence of type-I and type-II Fisher, we know that it also equals the GGN in the above scenarios.

\subsubsection{The Empirical Fisher (EF)}\label{sec:emp_fisher}
Finally, we introduce the empirical Fisher information matrix.
This matrix serves as an approximation to the Type-I Fisher that does not require Monte Carlo sampling.
Several contemporary optimizers, e.g.~the Adam variants, gained inspiration from second-order optimization, in particular preconditioning with the empirical Fisher.
For example, Adam stores the moving average of the mini-batch loss' squared gradients, which is motivated by (but is not equivalent to) preconditioning with the \emph{diagonal} of the empirical Fisher, defined below.\footnote{See~\cite{lin2024can} for the exact preconditioner Adam approximates, which is a different approximation of the Fisher information matrix.}

\begin{definition}[Empirical Fisher (\Cref{emp_fishers})]\label{def:emp_fisher}%
  The empirical matrix of the likelihood $\log r(\rvy \mid f(\vx_n, \vtheta))$,
  $\hat{\mF}(\vtheta) \in \sR^{D \times D}$ is defined as
  \begin{align*}
    & \hat{\mF}(\vtheta) \\
	& = \frac{1}{N} \sum_{n}
	\begin{aligned}[t]
	   & (-\nabla_{\vtheta} \log r(\rvy = \vy_n \mid \vf_n))        \\
	   & (-\nabla_{\vtheta} \log r(\rvy = \vy_n \mid \vf_n))^{\top} \\
	\end{aligned} \\
    & = \frac{1}{N} \sum_{n}
    \begin{aligned}[t]
       & \left(\jac_{\vtheta}\vf_n\right)^{\top}                  \\
       & (-\nabla_{\vf_n} \log r(\rvy = \vy_n \mid \vf_n))        \\
       & (-\nabla_{\vf_n} \log r(\rvy = \vy_n \mid \vf_n))^{\top} \\
       & \jac_{\vtheta}\vf_n.
    \end{aligned}
  \end{align*}
\end{definition}

\switchcolumn[1]*
\codeblock{basics/emp_fishers}
\codeblock{basics/emp_fisher_product}
\switchcolumn[0]

The subtle difference between the empirical and Type-I Fishers is that
$\mF^\text{I}(\vtheta)$ contains an \emph{expectation} over the gradient
outer product
\begin{align*}
  &(-\nabla_{\vtheta} \log r(\rvy = \vy_n \mid \vf_n)) \\
  &(-\nabla_{\vtheta} \log r(\rvy = \vy_n \mid \vf_n))^{\top},
\end{align*}
w.r.t.~the model's predictive distribution $r(\rvy \mid \vf_n)$,
whereas $\hat{\mF}(\vtheta)$ plugs in the \emph{ground-truth} labels $\vy_n$
into the gradient outer product. While the computation of the empirical Fisher
is more efficient than that Monte Carlo approximating the Type-I Fisher with
number of samples $M > 1$, this subtle difference
has severe implications for the utility of this matrix in optimization.
In particular,~\citet{kunstner2019limitations} show that preconditioning with
the empirical Fisher can have detrimental effects even in simple problems,
owing to its poor theoretical grounding. The empirical Fisher's success
in some settings (e.g., through Adam) can be attributed to its ability to attenuate
\emph{gradient noise}, not its properties as a curvature estimate.


%%% Local Variables:
%%% mode: latex
%%% TeX-master: "../main"
%%% End:
