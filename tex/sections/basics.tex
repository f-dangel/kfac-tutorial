\subsection{Empirical risk minimization \& Maximum Likelihood Estimation}

\subsubsection{Deep neural networks}

\subsubsection{Probabilistic interpretation}


\begin{align}
  \gL(\vtheta) &= \sum_{n=1}^N \ell(\vtheta, \vx_n, \vy_n)
  \\
               &=
                 \sum_{n=1}^N c(f(\vtheta, \vx_n), \vy_n)
\end{align}

\switchcolumn[0]

\blindtext

\switchcolumn[1]

\switchcolumn[0]* % sync

\blindtext

\subsection{Derivatives \& Automatic Differentiation}

\begin{caveat}
  In deep learning, we often work with matrices, or higher-dimensional tensors.
  We want to use matrix linear algebra expressions to avoid using heavy index notation.
  This can be achieved by flattening all tensors back into vectors and re-using definitions derivatives from the vector case.
  However, we must be careful when translating the results back to the tensor format, as the translation process depends on the flattening convention.
  Classically, the mathematical derivations prefer a \emph{different} flattening scheme than the one used in deep learning libraries.
\end{caveat}

\switchcolumn[0]*
\subsubsection{Flattening}
\switchcolumn[1]
\begin{example}[Matrix flattening, \Cref{basics/flattening}]\label{ex:flattening}
  For a matrix
  \begin{equation*}
    \mA = \begin{pmatrix} 1 & 2 \\ 3 & 4 \end{pmatrix}
  \end{equation*}
  we have
  \begin{equation*}
    \rvec(\mA)
    =
    \begin{pmatrix}
      1 \\ 2 \\ 3 \\ 4
    \end{pmatrix}\,,
    \qquad
    \cvec(\mA)
    =
    \begin{pmatrix}
      1 \\ 3 \\ 2 \\ 4
    \end{pmatrix}\,.
  \end{equation*}
\end{example}
\switchcolumn[0]

\vspace{\baselineskip}
\begin{caveat}[Flattening]
  In deep learning, we often work with matrices, or higher-dimensional tensors.
  We want to use matrix linear algebra expressions to avoid using heavy index notation.
  This can be achieved by flattening all tensors back into vectors and reusing definitions of derivatives from the vector case.
  However, we must be careful when translating the results back to the tensor format, as the translation process depends on the flattening convention.
  Classically, the mathematical derivations prefer a \emph{different} flattening scheme than the one used in deep learning libraries.
  This can cause confusion and bugs.
\end{caveat}

\switchcolumn[1]
\codeblock{basics/flattening}
\switchcolumn[0]

There are many ways to flatten the entries of a tensor into a vector.
The two by far most common conventions are (i) last-varies-fastest ($\rvec$) and (ii) first-varies-fastest ($\cvec$).
Their names are easy to remember from their action on a matrix (see \Cref{ex:flattening}): $\cvec$-flattening concatenates columns into a vector (column flattening); $\rvec$-flattening concatenates rows into a vector (row flattening).

Column-flattening is popular in mathematical presentations, while row-flattening is popular in deep learning libraries, which lay out tensors in row-major format in memory.
To see their differences, we will implement both (\Cref{basics/flattening}).
For arbitrary tensors, we can generalize the matrix flattenings by ordering entries such that either their first index ($\cvec$, \Cref{def:cvec}) or last index ($\rvec$, \Cref{def:rvec}) varies fastest:


\begin{setup}[Rank-$A$ tensor]\label{setup:flattening}
  Let $\tA \in \sR^{N_1 \times \dots \times N_A}$ be a tensor of rank $A$ whose entries are indexed through a tuple $(n_1, \dots, n_A)$ where $n_a \in \{1, \dots, N_a\}$ for $a \in \{1, \dots, A\}$.
  Vectors are rank-1 tensors, and matrices are rank-2 tensors.
\end{setup}
\begin{definition}[$\cvec$, \Cref{basics/flattening}]\label{def:cvec}
  The first-varies-fastest flattening of tensor $\tA$ from \Cref{setup:flattening} is
  \begin{align*}
    \cvec(\tA) =
    \begin{pmatrix}
      \etA_{\colored{1},1,\dots,1}   \\
      \etA_{\colored{2},1,\dots,1}   \\
      \vdots               \\
      \etA_{\colored{N_1},1,\dots,1} \\
      \etA_{\colored[VectorPink]{1},2,\dots,1}   \\
      \vdots               \\
      \etA_{\colored[VectorPink]{N_1},2,\dots,1} \\
      \vdots               \\
      \etA_{N_1,N_2,\dots,N_A}
    \end{pmatrix}
    \in \sR ^{N_1 \cdots N_A}\,.
  \end{align*}
\end{definition}

\begin{definition}[$\rvec$, \Cref{basics/flattening}]\label{def:rvec}
  The last-varies-fastest flattening of tensor $\tA$ from \Cref{setup:flattening} is
  \begin{align*}
    \rvec(\tA) =
    \begin{pmatrix}
      \etA_{1,\dots,1,\colored{1}}   \\
      \etA_{1,\dots,1,\colored{2}}   \\
      \vdots               \\
      \etA_{1,\dots,1,\colored{N_A}} \\
      \etA_{1,\dots,2,\colored[VectorPink]{1}}   \\
      \vdots               \\
      \etA_{1,\dots,2,\colored[VectorPink]{N_A}} \\
      \vdots               \\
      \etA_{N_1,\dots,N_{A-1},N_A}
    \end{pmatrix}
    \in \sR ^{N_A \cdots N_1}\,.
  \end{align*}
\end{definition}

In code, we will sometimes require partial flattening of a subset of contiguous indices, instead of all indices (\eg to turn a tensor into a matrix by first flattening the row indices, followed by flattening the column indices).
The definitions are analogous, but the flattened indices are surrounded by static ones.
%%% Local Variables:
%%% mode: latex
%%% TeX-master: "../main"
%%% End:


\switchcolumn[0]*
\subsubsection{Jacobians, JVP, VJPs}
Building up to curvature approximations that tackle the approximation of second-order partial derivatives, we start with first-order derivatives.
These are collected into a matrix called the Jacobian, which depends on the flattening convention.
We can multiply with the Jacobian and its transpose via automatic differentiation, without building up the matrix in memory.
These operations are called vector-Jacobian products (VJPs) and Jacobian-vector products (JVPs), respectively.
Machine learning libraries like JAX and PyTorch offer routines for computing Jacobians, VJPs, and JVPs.
However, their interface is functional.
Here, we provide an alternative implementation which accepts nodes of a computation graph rather than functions as input and will be beneficial for modular implementations of neural networks, as we consider later.
We also provide examples for important Jacobians, namely the output-parameter Jacobian of an affine map, i.e.\, a linear layer.
These Jacobians exhibit Kronecker structure, which is the foundation for the `K' in KFAC.
We also verify this structure numerically.
Crucially, the Kronecker structure changes depending on the flattening convention.

\begin{setup}[Vector-to-vector function]\label{setup:vector_to_vector_function}
  Let function $f: \sR^A \to \sR^B, \va \mapsto \vb = f(\va)$ denote a vector-to-vector function.
\end{setup}

\begin{definition}[Jacobian of a vector-to-vector function]\label{def:vector_jacobian}
  The Jacobian of a vector-to-vector function $f$ from \Cref{setup:vector_to_vector_function}, $\jac_{\va}\vb \in \sR^{B \times A}$, collects the first-order partial derivatives into a matrix such that
  \begin{align*}
    [\jac_{\va} \vb]_{i,j} = \frac{\partial [f(\va)]_i}{\partial [\va]_j}\,.
  \end{align*}
\end{definition}
\Cref{def:vector_jacobian} is limited to vector-to-vector functions.
The more general Jacobian of a tensor-to-tensor function can be indexed with combined indices from the input and output domain:

\begin{setup}[Tensor-to-tensor function]\label{setup:jacobians}
  Consider a tensor-to-tensor function $f: \sR^{A_1 \times \dots \times A_N} \to \sR^{B_1 \times \dots \times B_M}, \tA \mapsto \tB = f(\tA)$ from a rank-$N$ tensor $\tA$ into a rank-$M$ tensor $\tB$.
\end{setup}

\begin{definition}[General Jacobian, \Cref{jacobians}]\label{def:general_jacobian}
  The general Jacobian of $f$ from \Cref{setup:jacobians}, $\tJ_{\tB}\tA$, is a rank-$(M+N)$ tensor that collects the first-order partial derivatives such that
  \begin{align*}
    [\tJ_{\tA}\tB]_{j_1, \cdots ,j_M, i_1, \cdots, i_N} = \frac{\partial [f(\tA)]_{j_1, \cdots, j_M}}{\partial [\tA]_{i_1, \cdots, i_N}}\,.
  \end{align*}
\end{definition}
For $M=N=1$, the general Jacobian reduces to the Jacobian of a vector-to-vector function from \Cref{def:vector_jacobian}.

\paragraph{Multiplication} In practise, this general Jacobian can be prohibitively large and therefore one must almost always work with it in a matrix-free fashion, i.e.\, through VJPs and JVPs.

\switchcolumn[1]*
\codeblock{basics/jacobian_products}
\switchcolumn[0]

\begin{definition}[Vector-Jacobian products (VJPs), \Cref{jacobian_products}]\label{def:vjp}
  Given a tensor-to-tensor function $f$ from \Cref{setup:jacobians} and a tensor $\tV \in \sR^{B_1 \times \dots \times B_M}$ in the output domain, the vector-Jacobian product (VJP) $\tU$ of $\tV$ and $\tJ_{\tB}\tA$ lives in the $f$'s input domain and follows by contracting out the output indices,
  \begin{align*}
    &[\tU]_{i_1, \dots, i_N}
    \\
    &=
      \sum_{j_1, \dots, j_M}
      [\tV]_{j_1, \dots, j_M}
      [\tJ_{\tA}\tB]_{j_1, \dots, j_M, i_1, \dots, i_N}\,.
  \end{align*}
\end{definition}
For $M=N=1$, $\tV, \tU \to \vv, \vu$ are column vectors, $\tJ_{\tA}\tB \to \jac_{\va}\vb$ is a matrix, and the VJP is $\vu^{\top} = \vv^{\top} (\jac_{\va}\vb)$ or $\vu = (\jac_{\va}\vb)^{\top} \vv$, i.e.\,multiplication with the transpose Jacobian.

VJPs are at the heart of reverse-mode automatic differentiation, aka backpropagation (this is why $\tU$ is often called the \emph{pull-back} of $\tV$ through $f$).
Therefore, they are easy to implement with standard functionality.

The other popular contraction is between the Jacobian and a vector from the function's input domain that yields:

\begin{definition}[Jacobian-vector products (JVPs), \Cref{jacobian_products}]\label{def:jvp}
  Given a tensor-to-tensor function $f$ from \Cref{setup:jacobians} and a tensor $\tV \in \sR^{A_1 \times A_N}$ in the input domain, the Jacobian-vector product (JVP) $\tU$ between $\tV$ and $\tJ_{\tB}\tA$ lives in $f$'s output domain and follows by contracting the output indices,
  \begin{align*}
    &[\tU]_{j_1, \dots, j_M}
    \\
    &=
      \sum_{i_1, \dots, i_N}
      [\tJ_{\tA}\tB]_{j_1, \dots, j_M, i_1, \dots, i_N}
      [\tV]_{i_1, \dots, i_N}\,.
  \end{align*}
\end{definition}
For the vector case, $\tU, \tV, \tJ_{\tA}\tB \to \vu, \vv, \jac_{\va}\vb$, the JVP is $\vu = (\jac_{\va}\vb) \vv$, as suggested by its name.
JVPs are common in forward mode automatic differentiation ($\tU$ is often called the \emph{push-forward} of $\tV$ through $f$).
Only recently has this mode garnered attention.
Hence, the current JVP functionality in ML libraries usually follows a functional API.
To obtain an implementation that accepts variables from a computation graph and is more compatible with the modular approach we chose in this tutorial, we can use a trick\footnote{See \url{https://j-towns.github.io/2017/06/12/A-new-trick.html}.} that implements a VJP using two VJPs.

Jacobian products are efficient, but somewhat more abstract to work with as we cannot `touch' the full tensor. Also, we often also would like to think about this tensor as a matrix to be able to present derivations using linear algebra notation.

\switchcolumn[1]*
\codeblock{basics/jacobians}
\switchcolumn[0]

\paragraph{Matricization} We can reduce the general Jacobian back to the Jacobian matrix in two different ways: We can either directly matricize the tensor, or `flatten' the function $f \to f^{\vec}$ such that it consumes and produces vectors, then compute its Jacobian.
Both ways and their resulting Jacobian matrices depend on the flattening convention we choose.
The following definitions are consistent in the sense that both of the aforementioned approaches yield the same result, illustrated by this commutative diagram:

\begin{figure}[!h]
  \centering
  \begin{tikzpicture}[%
    font=\scriptsize,%
    thick,
    box/.style = {rectangle, draw=black, rounded corners, fill=VectorGray!50},%%
    ]
    \node[box] (A) at (0,0) {$f: \tA \mapsto \tB = f(\tA)$};
    \node[box] (B) at (4,0) {$\jac_{\tA}\tB$};
    \node[box, align=center] (C) at (0,-2.5) {%
      $f^{\vec} = \vec \circ f \circ \vec^{-1}$\\%
      $\vec(\tA) \coloneq \va \mapsto \vec(\tB) \coloneq \vb$%
    };
    \node[box, align=center] (D) at (4,-2.5) {%
      $\jac_{\va} \vb$\\%
      $=$\\%
      $\mat(\tJ_{\tA}\tB)$\\%
      $\coloneq$\\%
      $\jac^{\vec}_{\tA}\tB$%
    };
    \draw[-Stealth] (A.east) -- node[fill=white] {$\tJ$} (B.west);
    \draw[-Stealth] (A.south) -- node[fill=white] {flatten $f$} (C.north);
    \draw[-Stealth] (C.east) -- node[fill=white] {$\jac$} (D.west);
    \draw[-Stealth] (B.south) -- node[fill=white] {matricize $\jac_{\tA}\tB$} (D.north);
  \end{tikzpicture}
  \caption{Flattening and taking the Jacobian commute and lead to the same matricized Jacobian.
    $\vec$ denotes one of the flattening conventions from \Cref{def:cvec,def:rvec}.
    $\mat$ denotes matricization and involves two flattenings for the row and column dimensions, respectively.}
\end{figure}

Hence, there are two of interest for the purpose of this tutorial: the $\cvec$- and $\rvec$-Jacobians.
The $\cvec$-Jacobian is used in mathematical derivations in the literature.
The $\rvec$-Jacobian is common in code.

\begin{definition}[$\cvec$-Jacobian, \Cref{jacobians}]\label{def:cvec_jacobian}
  For a tensor-to-tensor function $f$ from \Cref{setup:jacobians}, its $\cvec$-Jacobian $\jac^{\cvec}_{\tA}\tB \in \sR^{A_N \cdots A_1 \times B_M \cdots B_1}$ results from flattening input and output tensors with $\cvec$ and applying the Jacobian definition for vectors,
  \begin{align*}
    [\jac^{\cvec}_{\tA}\tB]_{i,j}
    =
    \frac{\partial [\cvec(f(\tA))]_i}{\partial [\cvec(\tA)]_j}\,.
  \end{align*}
\end{definition}

\begin{definition}[$\rvec$-Jacobian, \Cref{jacobians}]\label{def:rvec_jacobian}
  For a tensor-to-tensor function $f$ from \Cref{setup:jacobians}, its $\rvec$-Jacobian $\jac^{\rvec}_{\tA}\tB \in \sR^{A_1 \cdots A_N \times B_1 \cdots B_M}$ results from flattening input and output tensors with $\rvec$ and applying the Jacobian definition for vectors,
  \begin{align*}
    [\jac^{\rvec}_{\tA}\tB]_{i,j}
    =
    \frac{\partial [\rvec(f(\tA))]_i}{\partial [\rvec(\tA)]_j}\,.
  \end{align*}
\end{definition}

As we will see in the following examples, the two Jacobians usually differ from each other, albeit in subtle ways. We highlight their differences on a linear layer, which will be useful later on when we discuss KFAC.

\switchcolumn[1]*
\codeblock{basics/jacobians_linear_layer}
\switchcolumn[0]

\begin{example}[$\cvec$- and $\rvec$-weight Jacobians of a linear layer, \Cref{jacobians_linear_layer}]\label{ex:weight_jacobians_linear_layer}
  Consider an affine map with weight matrix $\mW \in \sR^{D_{\text{out}} \times D_{\text{in}}}$, bias vector $\vb \in \sR^{D_{\text{out}}}$, input vector $\vx \in \sR^{D_{\text{in}}}$ and output vector $\vz \in \sR^{D_{\text{out}}}$ with
  \begin{align*}
    \vz
    \coloneqq
    \mW \vx + \vb
    =
    \begin{pmatrix}
      \mW & \vb
    \end{pmatrix}
    \begin{pmatrix}
      \vx \\ 1
    \end{pmatrix}
    \coloneqq
    \tilde{\mW}
    \tilde{\vx}\,.
  \end{align*}
  To express this operation as matrix-vector multiplication, we combined weight and bias into a single matrix $\tilde{\mW}$ and augment the input with a one, yielding $\tilde{\vx}$, to account for the bias contribution.

  The $\cvec$-Jacobian w.r.t.\,the combined weight is
  \begin{align*}
    \jac^{\cvec}_{\tilde{\mW}}\vz
    =
    \tilde{\vx}^{\top}
    \otimes
    \mI_{D_{\text{out}}}\,.
  \end{align*}
  In contrast, the $\rvec$-Jacobian is
  \begin{align*}
    \jac^{\cvec}_{\tilde{\mW}}\vz
    =
    \mI_{D_{\text{out}}}
    \otimes
    \tilde{\vx}^{\top}\,,
  \end{align*}
  see \Cref{jacobians_linear_layer}.
  Note that the order of Kronecker factors is \emph{reversed}, depending on the flattening scheme.
\end{example}

\begin{example}[$\cvec$- and $\rvec$-weight Jacobians of a linear layer with weight sharing]
  Consider the same affine map from above, but now processing multiple input vectors $\mX = \begin{pmatrix}\vx_1 & \dots & \vx_S\end{pmatrix} \in \sR^{D_{\text{in}}\times S}$, yielding a sequence $\mZ = \begin{pmatrix} \vz_1 & \dots & \vz_S\end{pmatrix} \in \sR^{D_{\text{out}}\times S}$ where each $\vz_s$ is produced like above.
  The parameters are \emph{shared} over all vectors in the input sequence.
  In matrix notation,
  \begin{align*}
    \mZ
    &\coloneqq
      \mW \mX + \vb \vone^{\top}_S
    \\
    &=
      \begin{pmatrix}
        \mW & \vb
      \end{pmatrix}
      \begin{pmatrix}
        \mX \\ \vone^{\top}_S
      \end{pmatrix}
      \coloneqq
      \tilde{\mW}
      \tilde{\mX}\,.
  \end{align*}
  The $\cvec$-Jacobian w.r.t.\,the combined weight is
  \begin{align*}
    \jac^{\cvec}_{\tilde{\mW}}\mZ
    =
    \tilde{\mX}^{\top}
    \otimes
    \mI_{D_{\text{out}}}\,.
  \end{align*}
  In contrast, the $\rvec$-Jacobian is
  \begin{align*}
    \jac^{\cvec}_{\tilde{\mW}}\mZ
    =
    \mI_{D_{\text{out}}}
    \otimes
    \tilde{\mX}^{\top}\,.
  \end{align*}
\end{example}

\switchcolumn[1]
\codeblock{basics/jacobians_shared_linear_layer}

%%% Local Variables:
%%% mode: latex
%%% TeX-master: "../main"
%%% End:


\switchcolumn[0]*
\subsubsection{Hessians, HVPs}
Now that we have covered first-order derivatives, we will move on to second-order derivatives and develop the necessary concepts to understand KFAC, as well as their implementation.
Second-order derivatives are collected into an object called \emph{the Hessian}.
For our purposes, it will be sufficient to consider the Hessian of functions producing a scalar-valued output.
Let's start with the definition of the Hessian of a vector-to-scalar function.

\begin{definition}[Hessian of a vector-to-scalar function]\label{def:vector_hessian}
  Let $\vf: \sR^A \to \sR, \va \mapsto b = f(\va)$ be a vector-to-scalar function.
  Its Hessian $\hess_{\va}b \in \sR^{A \times A}$ collects the second-order partial derivatives into a matrix such that
  \begin{align*}
    [\hess_{\va}b]_{i,j}
    &=
      \frac{\partial^2 b}{\partial [\va]_i \partial [\va]_j}\,.
  \end{align*}
\end{definition}
This definition is limited to functions with vector-valued arguments. The extension to tensor-to-scalar-functions is straightforward; however, it yields a tensor which is less convenient to work with in terms of notation:

\begin{setup}\label{setup:hessians}
  Consider a tensor-to-scalar function $f: \sR^{A_1 \times \dots \times A_N} \to \sR, \tA \mapsto b = f(\tA)$ from a rank-$N$ tensor $\tA$ into a scalar $b$.
\end{setup}

\begin{definition}[General Hessian of a tensor-to-scalar function, \Cref{hessians}]\label{def:general_hessian}
  The general Hessian of $f$ from \Cref{setup:hessians}, $\tH_{\tA}b \in \sR^{A_1 \times \dots \times A_N \times A_1 \times \dots \times A_N}$, collects the second-order partial derivatives of $f$ into a tensor with
  \begin{align*}
    &[\tH_{\tA}b]_{i_1, \dots, i_N, j_1, \dots, j_N}
      \\
    &=
      \frac{\partial^2 b}{\partial [\tA]_{i_1, \dots, i_N} \partial [\tA]_{j_1, \dots, j_N}}\,.
  \end{align*}
\end{definition}
Just like for Jacobians, the Hessian tensor is usually too large to be stored in memory.
Hence, one usually works with it implicitly through matrix-vector products, which can be done without computing the Hessian:

\switchcolumn[1]*
\codeblock{hessian_product}
\switchcolumn[0]

\begin{definition}[Hessian-vector products (HVPs), \Cref{hessian_product}]\label{def:hvp}
  Given a tensor-to-scalar function $f$ from \Cref{setup:hessians} and a tensor $\tV \in \sR^{A_1 \times \dots \times A_N}$ in the input domain, the Hessian-vector product (HVP) $\tU$ of $\tV$ with $\hess$ is the result of the contraction with one of the Hessian's input indices,
  \begin{align*}
    &[\tU]_{i_1, \dots, i_N}
    \\
    &=
      \sum_{j_1, \dots, j_N}
      [\tH_{\tA}b]_{i_1, \dots, i_N, j_1, \dots, j_N} [\tV]_{j_1, \dots, j_N}\,.
  \end{align*}
\end{definition}
For the vector case $N=1$, we have $\tV, \tA, \tH_{\tA}b \to \vv, \va, \hess_{\va}b$ and $\tU \to \vu = \hess_{\va} b$ as suggested by the name.

One way to multiply by the Hessian uses the so-called Pearlmutter trick~\cite{pearlmutter1994fast}.
It relies on the fact that multiplication with higher-order derivatives can be done by nested first-order differentiation.
Hence, multiplication with the Hessian can be done with two VJPs.

\switchcolumn[1]*
\codeblock{hessians}
\switchcolumn[0]

\paragraph{Matricization:} For notational convenience, we will also define matricized versions of the general Hessian from \Cref{def:general_hessian}; the $\cvec$-, and $\rvec$-Hessian. Just like for the Jacobians, it does not matter whether we first flatten the function's input space then compute the Hessian, or compute the general Hessian then matricize it:

\begin{figure}[!h]
  \centering
  \begin{tikzpicture}[%
    font=\scriptsize,%
    thick,
    box/.style = {rectangle, draw=black, rounded corners, fill=VectorGray!50},%%
    ]
    \node[box] (A) at (0,0) {$f: \tA \mapsto b = f(\tA)$};
    \node[box] (B) at (4,0) {$\tJ_{\tA}b$};
    \node[box, align=center] (C) at (0,-2.5) {%
      $f^{\vec} = f \circ \vec^{-1}$\\%
      $\vec(\tA) \coloneq \va \mapsto b$%
    };
    \node[box, align=center] (D) at (4,-2.5) {%
      $\hess_{\va} b$\\%
      $=$\\%
      $\mat(\tH_{\tA}b)$\\%
      $\coloneq$\\%
      $\hess^{\vec}_{\tA}b$%
    };
    \draw[-Stealth] (A.east) -- node[fill=white] {$\tJ$} (B.west);
    \draw[-Stealth] (A.south) -- node[fill=white] {flatten $f$} (C.north);
    \draw[-Stealth] (C.east) -- node[fill=white] {$\jac$} (D.west);
    \draw[-Stealth] (B.south) -- node[fill=white] {matricize $\tH_{\tA}b$} (D.north);
  \end{tikzpicture}
  \caption{Flattening and taking the Hessian commute and lead to the same matricized Hessian.
    $\vec$ denotes one of the flattening conventions from \Cref{def:cvec,def:rvec}.
    $\mat$ denotes matricization and involves two flattenings for the row and column dimensions, respectively.}
\end{figure}


\begin{definition}[$\cvec$-Hessian]\label{def:cvec_hessian}
  For a tensor-to-scalar function $f$ from \Cref{setup:hessians}, the $\cvec$-Hessian $\hess_{\tA}^{\cvec}b \in \sR^{A_N \cdots A_1 \times A_N \cdots A_1}$ results from flattening the input tensor with $\cvec$ and applying the Hessian from \Cref{def:vector_hessian},
  \begin{align*}
    [\hess^{\cvec}_{\tA}b]_{i, j}
    &=
      \frac{\partial^2 b}{\partial [\cvec(\tA)]_{i} \partial [\cvec(\tA)]_{j}}\,.
  \end{align*}
\end{definition}

\begin{definition}[$\rvec$-Hessian]\label{def:rvec_hessian}
  For a tensor-to-scalar function $f$ from \Cref{setup:hessians}, the $\rvec$-Hessian $\hess_{\tA}^{\rvec}b \in \sR^{A_1 \cdots A_N \times A_1 \cdots A_N}$ results from flattening the input tensor with $\rvec$ and applying the Hessian from \Cref{def:vector_hessian},
  \begin{align*}
    [\hess^{\rvec}_{\tA}b]_{i, j}
    &=
      \frac{\partial^2 b}{\partial [\rvec(\tA)]_{i} \partial [\rvec(\tA)]_{j}}\,.
  \end{align*}
\end{definition}

Whenever we consider vector-to-scalar functions, both Hessians are identical and we thus suppress the flattening scheme and write $\hess_{\va}b$. Let's look at some important examples of Hessians that we will come back to later in the text.

\switchcolumn[1]*
\codeblock{hessian_ce_loss}
\switchcolumn[0]

\begin{example}[Softmax cross-entropy loss Hessian, \Cref{hessian_ce_loss}]
  Consider the softmax cross-entropy loss function between the vector-valued logits $\vx \in \sR^C$ and a class label $y \in \{1, \dots, C\}$:
  \begin{align*}
    \ell(\vx, y)
    &=
      -\log([p(\vx)]_y)\,.
  \end{align*}
  with $p(\vx) = \softmax(\vx) \in \sR^C$.
  Its Hessian w.r.t.\,$\vx$ is diagonal-minus-rank-one,
  \begin{align*}
    \hess_{\vx} \ell(\vx, y)
    =
    \diag(p(\vx)) - p(\vx) p(\vx)^\top\,.
  \end{align*}
  See for instance~\cite{dangel2020modular} for a derivation.
\end{example}

\switchcolumn[1]*
\codeblock{hessian_mse_loss}

\switchcolumn[0]
\begin{example}[Square loss Hessian, \Cref{hessian_mse_loss}]
  Consider the square loss function between a vector-valued input $\vx \in \sR^C$ and its associated target $\vy \in \sR^C$:
  \begin{align*}
    \ell(\vx, \vy)
    &=
      \left\lVert
      \vx - \vy
      \right\rVert^2
      \\
      &=
      (\vx - \vy)^{\top} \mI (\vx - \vy)\,.
  \end{align*}
  Its Hessian w.r.t.\,$\vx$ is proportional to the identity,
  \begin{align*}
    \hess_{\vx} \ell(\vx, \vy)
    =
    2 \mI_C\,.
  \end{align*}
\end{example}

We list more important loss function Hessians in \Cref{app:loss_function_hessians}.

%%% Local Variables:
%%% mode: latex
%%% TeX-master: "../main"
%%% End:


\switchcolumn[0]*
\subsection{Curvature Matrices in Deep Learning}
\subsubsection{The Hessian}

\subsubsection{The Generalized Gauss-Newton (GGN)}
Linearization

\switchcolumn[1]*
\codeblock{ggn_rosenbrock}
\switchcolumn[0]

\begin{example}[GGN for the Rosenbrock function, \Cref{ggn_rosenbrock}]
  Consider the 2d Rosenbrock function $f_{\alpha}: \sR^2 \to \sR$ with
  \begin{align*}
    f_{\alpha}(\vx)
    =
    (1 - x_1)^2 + \alpha (x_2 - x_1^2)^2\,.
  \end{align*}
  with some $\alpha > 0$.
  We can express $f_{\alpha} = \ell \circ g_{\alpha}$,
  \begin{align*}
    g_{\alpha}(\vx) = \begin{pmatrix}
                        x_1 \\
                        \sqrt{\alpha} (x_2 - x_1^2)
                      \end{pmatrix}
    \shortintertext{and convex}
    \ell(\vg) = \vg^\top \vg\,,
  \end{align*}
  namely square loss.

  Linearizing $g_{\alpha}$ w.r.t. $\vx$ around $\vx'$ gives
  \begin{align*}
    g^{\text{lin}}_{\alpha}(\vx) = g_{\alpha}(\vx') + (\jac_{\vx'}g_{\alpha}(\vx')) (\vx - \vx')\,.
  \end{align*}
  with
  \begin{align*}
    \jac_{\vx}g_{\alpha}(\vx)
    =
    \begin{pmatrix}
      1 & 0 \\
      -2 \sqrt{\alpha} x_1 & \sqrt{\alpha}
    \end{pmatrix}\,.
  \end{align*}
  The first way to form the Hessian is to take the Hessian of $f_{\alpha}' = \ell \circ g_{\alpha}^{\text{lin}}$, and evaluate it at $\vx' = \vx$,
  \begin{align*}
    (\hess_{\vx} f_{\alpha}'(\vx))|_{x=x'}
    \\
    =
    2 (\jac_{\vx} g_{\alpha}(\vx))|_{x=x'}^{\top}
    (\jac_{\vx} g_{\alpha}(\vx))|_{x=x'}
    \\
    =
    2
    \begin{pmatrix}
      1 + 4\alpha x_1^2& -2 \sqrt{\alpha} x_1\\
      -2 \sqrt{\alpha} x_1 & \alpha
    \end{pmatrix}
  \end{align*}
  This matrix is PSD because it is an outer product of two matrices.
  It is also different from the Hessian matrix
  \begin{align*}
    \hess_{\vx} f_{\alpha}(\vx)
    =
    \begin{pmatrix}
      2 + 12 \alpha x_1^2 - 4 \alpha x_2 & -4 \alpha x_1 \\
      -4 \alpha x_1 & 2 \alpha
    \end{pmatrix}
  \end{align*}
  \begin{figure}[H]
    \centering
    \includegraphics[width=\linewidth]{../kfs/plots/linearized_rosenbrock.pdf}
    \caption{The 2d Rosenbrock function $f_{\alpha=10}$ (solid contour lines) with and its approximation $f'_{\alpha=10, \vx'} = \ell \circ g^{\text{lin}}_{\alpha=10, \vx'}$ (dashed contour lines) when partially linearizing it around an anchor $\vx'$ (star).}
  \end{figure}
\end{example}

\subsubsection{The Fisher}
Probabilistic perspective

Explain type-1 versus type-2
\subsubsection{The Connection between GGN \& Fisher}
\subsubsection{The Empirical Fisher (EF)}

%%% Local Variables:
%%% mode: latex
%%% TeX-master: "../main"
%%% End:
