\subsection{Empirical risk minimization \& Maximum Likelihood Estimation}

\begin{paracol}{2}

  \begin{align}
    \gL(\vtheta) &= \sum_{n=1}^N \ell(\vtheta, \vx_n, \vy_n)
    \\
                 &=
                   \sum_{n=1}^N c(f(\vtheta, \vx_n), \vy_n)
  \end{align}

  \switchcolumn[0]

  \blindtext

  \switchcolumn[1]

  \switchcolumn[0]* % sync

  \blindtext

  \switchcolumn[1]

  \codeblockWithOutput{hello_world}

  \switchcolumn[0]

  \Cref{hello_world} says hello world.

  \subsection{Derivatives}

  \begin{caveat}
    In deep learning, we often work with matrices, or higher-dimensional tensors.
    We want to use matrix linear algebra expressions to avoid using heavy index notation.
    This can be achieved by flattening all tensors back into vectors and re-using definitions derivatives from the vector case.
    However, we must be careful when translating the results back to the tensor format, as the translation process depends on the flattening convention.
    Classically, the mathematical derivations prefer a \emph{different} flattening scheme than the one used in deep learning libraries.
  \end{caveat}

  \switchcolumn[0]*
  \subsubsection{Flattening}

  There are two popular flattening conventions: last-varies-fastest and first-varies-fastest.

  \begin{example}
    For matrix $\mA = \begin{pmatrix} 1 & 2 \\ 3 & 4 \end{pmatrix} $, we have
    \begin{equation*}
      \rvec(\mA)
      =
      \begin{pmatrix}
        1 \\ 2 \\ 3 \\ 4
      \end{pmatrix}\,,
      \qquad
      \cvec(\mA)
      =
      \begin{pmatrix}
        1 \\ 3 \\ 2 \\ 4
      \end{pmatrix}\,,
    \end{equation*}
    see \Cref{flattening}.
  \end{example}

  \switchcolumn[1]
  \codeblockWithOutput{flattening}

  \switchcolumn[0]*
  \subsubsection{Jacobians, JVP, VJPs}

  \begin{definition}[Jacobian]
    Consider a vector-to-vector function $f: \sR^A \to \sR^B, \va \mapsto \vb = f(\va)$.
    Its Jacobian $\jac_{\va}\vb \in \sR^{\mB \times \mA}$ collects the first-order partial derivatives into a matrix such that
    \begin{align*}
      [\jac_{\va} \vb]_{i,j} = \frac{\partial [f(\va)]_i}{\partial [\va]_j}\,.
    \end{align*}
  \end{definition}

  \begin{definition}[General Jacobian]
    TODO
  \end{definition}

  \begin{definition}[Vector-Jacobian products]
    TODO
  \end{definition}

  \begin{definition}[$\cvec$-Jacobian]
    Consider a tensor-to-tensor function $f: \sR^{A_1 \times \dots \times A_N} \to \sR^{B_1 \times \dots \times B_M}, \tA \mapsto \tB = f(\tA)$. Its $\cvec$-Jacobian $\jac^{\cvec}_{\tA}\tB \in \sR^{A_N \cdots A_1 \times B_M \cdots B_1}$ results from flattening input and output tensors with $\cvec$ and applying the Jacobian definition for vectors,
    \begin{align*}
      [\jac^{\cvec}_{\tA}\tB]_{i,j}
      =
      \frac{\partial [\cvec(f(\tA))]_i}{\partial [\cvec(\tA)]_j}\,.
    \end{align*}
  \end{definition}

  \begin{definition}[$\rvec$-Jacobian]
    Consider a tensor-to-tensor function $f: \sR^{A_1 \times \dots \times A_N} \to \sR^{B_1 \times \dots \times B_M}, \tA \mapsto \tB = f(\tA)$. Its $\cvec$-Jacobian $\jac^{\cvec}_{\tA}\tB \in \sR^{A_1 \cdots A_N \times B_1 \cdots B_M}$ results from flattening input and output tensors with $\cvec$ and applying the Jacobian definition for vectors,
    \begin{align*}
      [\jac^{\cvec}_{\tA}\tB]_{i,j}
      =
      \frac{\partial [\cvec(f(\tA))]_i}{\partial [\cvec(\tA)]_j}\,.
    \end{align*}
  \end{definition}

  \switchcolumn[1]
  \codeblockWithOutput{jacobians}


  \switchcolumn[1]*
  \codeblockWithOutput{jacobians_linear_layer}
  \switchcolumn[0]

  \begin{example}[$\cvec$- and $\rvec$-weight Jacobians of a linear layer]
    Consider an affine map with weight matrix $\mW \in \sR^{D_{\text{out}} \times D_{\text{in}}}$, bias vector $\vb \in \sR^{D_{\text{out}}}$, input vector $\vx \in \sR^{D_{\text{in}}}$ and output vector $\vz \in \sR^{D_{\text{out}}}$ with
    \begin{align*}
      \vz
      \coloneqq
      \mW \vx + \vb
      =
      \begin{pmatrix}
        \mW & \vb
      \end{pmatrix}
      \begin{pmatrix}
        \vx \\ 1
      \end{pmatrix}
      \coloneqq
      \tilde{\mW}
      \tilde{\vx}\,.
    \end{align*}
    To express this operation as matrix-vector multiplication, we combined weight and bias into a single matrix $\tilde{\mW}$ and augment the input with a one, yielding $\tilde{\vx}$, to account for the bias contribution.

    The $\cvec$-Jacobian w.r.t.\,the combined weight is
    \begin{align*}
      \jac^{\cvec}_{\tilde{\mW}}\vz
      =
      \tilde{\vx}^{\top}
      \otimes
      \mI_{D_{\text{out}}}\,.
    \end{align*}
    In contrast, the $\rvec$-Jacobian is
    \begin{align*}
      \jac^{\cvec}_{\tilde{\mW}}\vz
      =
      \mI_{D_{\text{out}}}
      \otimes
      \tilde{\vx}^{top}\,,
    \end{align*}
    see \Cref{jacobians_linear_layer}.
    Note that the order of Kronecker factors is \emph{reversed}, depending on the flattening scheme.
  \end{example}

  \begin{example}[$\cvec$- and $\rvec$-weight Jacobians of a linear layer with weight sharing]
    Consider the same affine map from above, but now processing multiple input vectors $\mX = \begin{pmatrix}\vx_1 & \dots & \vx_S\end{pmatrix} \in \sR^{D_{\text{in}}\times S}$, yielding a sequence $\mZ = \begin{pmatrix} \vz_1 & \dots & \vz_S\end{pmatrix} \in \sR^{D_{\text{out}}\times S}$ where each $\vz_s$ is produced like above.
    The parameters are \emph{shared} over all vectors in the input sequence.
    In matrix notation,
    \begin{align*}
      \mZ
      \coloneqq
      \mW \mX + \vb \vone^{\top}_S
      =
      \begin{pmatrix}
        \mW & \vb
      \end{pmatrix}
      \begin{pmatrix}
        \mX \\ \vone^{\top}_S
      \end{pmatrix}
      \coloneqq
      \tilde{\mW}
      \tilde{\mX}\,.
    \end{align*}
    The $\cvec$-Jacobian w.r.t.\,the combined weight is
    \begin{align*}
      \jac^{\cvec}_{\tilde{\mW}}\mZ
      =
      \tilde{\mX}
      \otimes
      \mI_{D_{\text{out}}}\,.
    \end{align*}
    In contrast, the $\rvec$-Jacobian is
    \begin{align*}
      \jac^{\cvec}_{\tilde{\mW}}\mZ
      =
      \mI_{D_{\text{out}}}
      \otimes
      \tilde{\mX}\,,
    \end{align*}
    see \Cref{jacobians_shared_linear_layer}.
  \end{example}

  \switchcolumn[1]
  \codeblockWithOutput{jacobians_shared_linear_layer}

\end{paracol}
\subsubsection{Hessians, HVPs}

\subsection{Curvature Matrices}
\subsubsection{The Generalized Gauss-Newton (GGN)}
\subsubsection{The Fisher}
\subsubsection{The Connection between GGN \& Fisher}
\subsubsection{The Empirical Fisher (EF)}

%%% Local Variables:
%%% mode: latex
%%% TeX-master: "../main"
%%% End:
